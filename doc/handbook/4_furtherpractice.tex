\section{Further Practice}
\label{tutorial-furtherpractice}

If there is a need for further practice, we refer here to the test problems that
are already implemented in \Dumux. Several examples for all models
can be found in the \texttt{test}-directory. %An overview over the available test
%cases can be found in the class documentation \url{http://www.dumux.org/documentation.php}.

Another possibility to gain more experience with \Dumux is the \texttt{dumux-lecture} module
that contains different application examples that are used in the lectures at the
Department of Hydromechanics and Modelling of Hydrosystems in Stuttgart.
The \texttt{dumux-lecture} module can be obtained with the \texttt{installExternal.py} 
script and the argument \texttt{lecture}. 

The module is structured based on the different lectures:
\begin{itemize}
\item mm: Multiphase Modelling,
\item efm: Environmental Fluid Mechanics,
\item mhs: Modelling of Hydrosystems.
\end{itemize}
The majority of applications are covered in the course Multiphase Modelling (mm),
while there are also some basic examples in the
courses Environmental Fluid Mechanics (efm) and Modelling of Hydrosystems (mhs).
These applications are primarily designed to enhance the understanding of conceptualizing the
governing physical processes and their implementation in a numerical simulator.
Different aspects of modeling multi-phase multi-component flow and transport processes are shown.
The lectures focus on questions such as the assignment of boundary conditions, the choice of the
appropriate physics for a given problem (which phases, which components), discretization issues,
time stepping. You can find, e. g., a comparison of different two-phase flow problems: The
simpler approach considers two immiscible fluids while components in both phases with inter-phase
mass transfer are considered in the more complex approach.
All scenarios and their physical background are explained in additional .tex-files,
which are provided in sub-directories named \texttt{description}. The following test cases are
contained in the \texttt{dumux-lecture} module:
\begin{itemize}
\item \texttt{buckleyleverett}: The Buckley-Leverett Problem is a classical porous media flow show case
\item \texttt{co2plume}: Analysis of the influence of the gravitational number on a $\text{CO}_2$ plume
\item \texttt{columnxylene}: An experiment of the Research Facility for Subsurface Remediation, University of Stuttgart
\item \texttt{convectivemixing}: A test case related to CO$_2$ storage
\item \texttt{fractures}: Two-phase flow in fractured porous media
\item \texttt{fuelcell}: Water management in PEM fuel cells 
\item \texttt{heatpipe}: A show case for two-phase two-component flow with heat fluxes
\item \texttt{heavyoil}: Steam assisted gravity drainage (SAGD)
\item \texttt{henryproblem}: A show case related to salt water intrusion
\item \texttt{mcwhorter}: The McWhorter Problem is a classical porous media flow show case
\item \texttt{naplinfiltration}: Infiltration of non-aqueous phase liquid (NAPL) into soil
\item \texttt{remediationscenarios}: Test case for NAPL contaminated unsaturated soils
\item \texttt{groundwater}: Simple groundwater flow case for the course Modelling of Hydrosystems (mhs)
\item Different single/two-phase, single/two-component problems: Examples from the course Environmental Fluid Mechanics (efm)
\end{itemize}
