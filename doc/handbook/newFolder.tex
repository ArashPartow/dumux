\section{Setup of a New Folder}

In this section setting up a new folder is described. In fact it is easy to create a new folder, but getting the build system to know the new folder takes some steps:

\begin{enumerate}[1)]
 \item create new folder with content
 \item adapt \verb+Makefile.am+
 \item insert new folder in \verb+Makefile.am+ of the directory above
 \item adapt \verb+configure.ac+ in the \verb+$DUMUX_ROOT+ (the directory you checked out, probably dumux)
 \item rerun dunecontrol / autogen for \Dumux
\end{enumerate}

\noindent In more detail:

\textbf{First} of all, the new folder including all relevant files needs to be created (see Section \ref{tutorial-coupled} and \ref{tutorial-decoupled} for description of a problem). 

\textbf{Second}, a new \verb+Makefile.am+ for the new Folder needs to be created. It is good practice to simply copy an existing file. For example the file \verb+$DUMUX_ROOT/test/2p/Makefile.am+ looks as follows:
\begin{verbatim}
bin_PROGRAMS = test_2p

test_2p_SOURCES = test_2p.cc
test_2p_CXXFLAGS = $(MPI_CPPFLAGS) 
test_2p_LDADD = $(MPI_LDFLAGS) 

include $(top_srcdir)/am/global-rules
\end{verbatim}

All occurrences of \verb+test_2p+ need to be replaced by the name of the new project, e.g. \verb+New_Project+. At least if the name of the source file as well as the name of the new project are \verb+New_Project+.

\textbf{Third}: In the directory above your new Project there is also a \verb+Makefile.am+ . In this file the subdirectories are listed. As you introduced a new subdirectory, it needs to be included here. In this case the name of the new Folder is \verb+New_Project+ . Don't forget the trailing backslash.

\begin{verbatim}
 SUBDIRS = . \
	  1p \
	  1p2c \
	  2p \
	  2p2c \
	  2p2cni \
	  2pni \
	New_Project \
...
\end{verbatim}

\textbf{Fourth}: In \verb+$DUMUX_ROOT+ there is a file \verb+configure.ac+. In this file, the respective Makefiles are listed. After a line reading

 \verb+AC_CONFIG_FILES([Makefile+ 

 \noindent a line, declaring a new Makefile, needs to be included. The Makefile itself will be generated automatically during the autogen run which is triggered by dunecontrol. For keeping track of the included files, inserting in alphabetical order is good practice. The new line could read: \verb+test/New_Project/Makefile+ 

\textbf{Fifth}: Recreate the build system by running dunecontrol as described in Section \ref{install}.

\paragraph{Committing a new folder to the Subversion repository}
For those who work with Subversion (svn) and want to commit a newly setup folder to the repository, additional guidelines apply:

\begin{enumerate}[1)]
 \item use svn attributes to ignore files which are automatically created by a dunecontrol run
 \item test \texttt{make headercheck}
 \item test Doxygen
\end{enumerate}

\noindent In more detail:

\textbf{First}: The command \verb+svn status+ marks all files which are not under version control with a question mark. Because dunecontrol creates a lot of files automatically this output becomes crowded and one might overlook ``real'' files which have to be added (they also will not be shown by a \verb+svn status -q+).
For the stable part of \Dumux there is the rule to ignore and only to ignore the folder {\em .deps}, executables \texttt{test\_*}, and the files {\em Makefile} and {\em Makefile.in}.

How to set the svn attributes:
\begin{itemize}
 \item{\em eclipse}: right click on the file/folder $\rightarrow$ ``team'' $\rightarrow$ ``add to svn:ignore\dots''
 \item{\em kdesvn}: right click on the file/folder $\rightarrow$ ``ignore/unignore current item''
 \item{\em svn on shell}: \verb+svn propset svn:ignore FILETOIGNORE .+
\end{itemize}
Commit the changes for example in the command line with \verb+svn commit -m+ to the repository. It is also possible to use wildcards, e.\,g., if you want to ignore all vtu-files in your application folder in dumux-devel set \verb+FILETOIGNORE+ to \verb+'*.vtu'+. Remember that such an ignore is only allowed in dumux-devel and not in the stable dumux.

\textbf{Second}: It is good practice that every header includes everything it uses by itself and does not rely on the includes from headers that are included by themselves. This can be tested with \texttt{make headercheck} and should be done before committing to stable dumux. To test a specific header, use \texttt{make} \texttt{headercheck} \texttt{HEADER=} and add the header's path.

\textbf{Third}: Run \texttt{make doc} in the \texttt{\$DUMUX\_ROOT} directory to generate the class documentation by Doxygen. Have a look into \texttt{\$DUMUX\_ROOT/doc/doxygen/doxyerror.log} whether one of your files have errors or are causing warnings. Please fix them.
