\section{Assembling the linear system}
\label{sc_linearsystem}
The physical system is implemented as the mathematical differential equation in
local operators. \Dumux generates the linear system automatically. Read on, to
learn what is done internally.

% \subsection{Newton's algorithm}
The differential equations are implemented in the residual form. All terms are
on the left hand side and are summed up. The terms contains values for the primary
variables which are part of the solution vector $\textbf{u}$. The sum of the terms
is called residual $\textbf{r}(\textbf{u})$ which is a function of the solution. For
example:
\begin{align*}
\underbrace{
  \phi \frac{\partial \varrho_\alpha S_\alpha}{\partial t}
 -
 \text{div} \left(
 \varrho_\alpha \frac{k_{r\alpha}}{\mu_\alpha} \mbox{\bf K}
 \left(\grad\, p_\alpha - \varrho_{\alpha} \mbox{\bf g} \right)
 \right) - q_\alpha} _
{\textbf{r}(\textbf{u})}
= 0
\end{align*}

We don't know the solution $\textbf{u}$, so we use the iterative Newton algorithm to
obtain a good estimate of $\textbf{u}$. We start with an initial guess $\textbf{u}^0$ and
calculate it's residual $\textbf{r}(\textbf{u}^0)$. To minimize the error, we calculate
the derivative of the residual with respect to the solution. This is the Jacobian
matrix
\begin{align*}
  \frac{\text{d}}{\text{d}\textbf{u}}\textbf{r}(\textbf{u}^i)
  = J_{\textbf{r}(\textbf{u}^i)}
  = \left(\frac{\text{d}}{\text{d}\textbf{u}^i_m}\textbf{r}(\textbf{u}^i)_n\right)_{m,n}
\end{align*}
with $i$ denoting the Newton iteration step.
Each column is the residual derived with respect to the $m$th entry of $\textbf{u}^i$.

The Jacobian indicates the direction where the residual increases. By solving the
linear system
\begin{align*}
  J_{\textbf{r}(\textbf{u}^i)} \cdot \textbf{x}^i = \textbf{u}^i
\end{align*}
we calculate the direction of maximum growth $\textbf{x}^i$. We subtract it from
our current solution to get a new, better solution
$\textbf{u}^{i+1} = \textbf{u}^i - \textbf{x}^i$.

We repeat the calculation of of the Jacobian $J_{\textbf{r}(\textbf{u}^i)}$ and the
direction of maximum growth $\textbf{x}^i$ until our solution becomes good enough.
