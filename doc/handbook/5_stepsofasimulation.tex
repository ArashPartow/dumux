\section{Steps of a \Dumux Simulation}
\label{flow}

\todo[inline]{Hier könnte man den Text auch etwas abspecken}
This chapter is supposed to show how things are ``handed around'' in \Dumux. This
is not a comprehenisve guide through the modeling framework of \Dumux, but
hopefully it will help getting to grips with it.

In Section \ref{content} the structure of \Dumux is shown from a \emph{content}
point of view.
Section \ref{implementation} is written from the point of view of the \emph{implementation}.
These two approaches are linked by the circled numbers (like \textbf{\textcircled{\ref{init}}})
in the flowchart of Section \ref{implementation} corresponding to the enumeration
of the list of Section \ref{content}. This is supposed to demonstrate at which point
of the program-flow you are content- and implementation-wise.

Section \ref{implementation} is structured by \fbox{boxes} and
$\overrightarrow{\textnormal{arrows}}$. Boxes stand for more or less important
points in the programm. They may may be reckoned ``step stones''. Likewise, the
arrows connect the boxes. If important things happen in between, it is written
under the arrows.

\fbox{Plain boxes} stand for generic parts of the program. \fbox{\fbox{double}}
$\lbrace\lbrace$boundings$\rbrace\rbrace$ stand for the implementation specific
part of the program, like \verb+2p, 2p2c...+. This will be the most important
part for most users. \uwave{snakelike lines} tell you that this part is specific
to the components considered.

For keeping things simple, the program flow of a \verb+2p+ model is shown.
There are extensive comments regarding the formating in the tex file: so feel free,
to enhance this description.

\subsection{Structure -- by Content}

\label{content}
% by means of this enumerated list, the connection between algorithm and content
% can be achieved by references to the labels of this list.
This list shows the algorithmic outline of a typical \Dumux run. Each item stands
for a characteristic step within the modeling framework.

\clearpage
In Figure \ref{fig:algorithm}, the algorithmic representations of both approaches
down to the element level are illustrated.

\begin{figure}[hbt]
\begin{tabular}{ l | l }

\begin{minipage}[t]{0.48\textwidth}
\setcounter{thingCounter}{0}

\scriptsize
\sffamily
\begin{tabbing}
\textbf{{\begin{turn}{45}\color{black}\numberThis{main}{init}\end{turn}}}             \=
\textbf{{\begin{turn}{45}\color{dumuxBlue}\numberThis{time step}{prep}\end{turn}}}            \=
\textbf{{\begin{turn}{45}\color{Mulberry}\numberThis{\textsc{Newton}}{elem}\end{turn}}}         \=
\textbf{{\begin{turn}{45}\color{dumuxYellow}\numberThis{element}{calc}\end{turn}}}             \=  \\
\\
\color{black}initialize \\
\color{black}\textbf{foreach} time step\\

  \> \color{dumuxBlue}prepare update\\
  \> \color{dumuxBlue}\textbf{foreach} \textsc{Newton} iteration \\

    \> \> \color{Mulberry}\textbf{foreach} element \\

      \> \> \> \color{dumuxYellow}- calculate element \\
      \> \> \> \color{dumuxYellow}\; residual vector and \\
      \> \> \> \color{dumuxYellow}\; Jacobian matrix\\
      \> \> \> \color{dumuxYellow}- assemble into global\\
      \> \> \> \color{dumuxYellow}\; residual vector and \\
      \> \> \> \color{dumuxYellow}\;{Jacobian} matrix \\

    \> \> \color{Mulberry}\textbf{endfor} \\

    \> \> \color{Mulberry}solve linear system\\
    \> \> \color{Mulberry}update solution\\
    \> \> \color{Mulberry}check for \textsc{Newton} convergence\\
  \> \color{dumuxBlue}\textbf{endfor}\\
  \> \color{dumuxBlue}- adapt time step size, \\
  \> \color{dumuxBlue}\; possibly redo with smaller step size\\
  \> \color{dumuxBlue}- write result\\
\color{black}\textbf{endfor}\\
\color{black}finalize
\end{tabbing}

\end{minipage}

&

\begin{minipage}[t]{0.48\textwidth}
\setcounter{thingCounter}{0}

\scriptsize
\sffamily
\begin{tabbing}
\textbf{{\begin{turn}{45}\color{black}1. main\end{turn}}}             \=
\textbf{{\begin{turn}{45}\color{dumuxBlue}2. time step\end{turn}}}            \=
\textbf{{\begin{turn}{45}\color{Mulberry}3. \textsc{IMPES/C}\end{turn}}}        \=
\textbf{{\begin{turn}{45}\color{dumuxYellow}4. element\end{turn}}}             \=  \\
\\
\color{black}initialize \\
\color{black}\textbf{foreach} time step\\

  \> \color{dumuxBlue}prepare update\\
  \> \color{dumuxBlue}\textbf{foreach} \textsc{IMPES/C} step \\
    \> \> \color{Mulberry}\textbf{if} grid is adaptive\\
      \> \> \> \color{dumuxYellow}- calculate refinement indicator\\
      \> \> \> \color{dumuxYellow}- mark elements, adapt the grid\\
      \> \> \> \color{dumuxYellow}- map old solution to new grid\\
    \> \> \color{Mulberry}- calculate {flow field}\\
    \> \> \color{Mulberry}\textbf{foreach} element \\

      \> \> \> \color{dumuxYellow}- calculate element stiffness matrix \\
      \> \> \> \color{dumuxYellow}- assemble into global matrix \\

    \> \> \color{Mulberry} \textbf{endfor} \\
    \> \> \color{Mulberry} solve linear system\\

    \> \> \color{Mulberry}- calculate {transport} \\
    \> \> \color{Mulberry}\; (saturations, concentrations,...) \\
    \> \> \color{Mulberry}\textbf{foreach} element  \\
      \> \> \> \color{dumuxYellow}-calculate update (explicitly) \\
      \> \> \> \color{dumuxYellow}- adapt time step ({CFL}-like criterion) \\
    \> \> \color{Mulberry}\textbf{endfor} \\
    \> \> \color{Mulberry}- update old solution \\
    \> \> \color{Mulberry}- postprocess (flash calculation, etc.)\\
  \> \color{dumuxBlue}\textbf{endfor}\\
  \> \color{dumuxBlue}- write result\\
\color{black}\textbf{endfor}\\
finalize
\end{tabbing}

\end{minipage}
\end{tabular}

\caption{Structure of a coupled fully-implicit (\textbf{left}) and a decoupled
semi-implicit (\textbf{right}) scheme in \Dumux.}
\label{fig:algorithm}
\end{figure}

\subsection{Structure --  by Implementation}
 \label{implementation}
This section is supposed to help you in getting an idea how things are handled in
\Dumux and in which files things are written down.
This is not intuitivly clear, therefore it is mentioned for each \fbox{step-stone}.
\textbf{called by} tells you from which file a function is
accessed. \textbf{implemented in} tells you in which file the function is written
down. The name of the function is set in \verb+typewriter+.
Being a function is indicated by round brackets \verb+()+ but only the function
name is given and not the full signature (arguments...) .
Comments regarding the events within one step-stone are set \scriptsize{smaller}.

\begin{landscape}
\pagestyle{empty} % switch off headings and footer in order to get more space for the flowchart
\setlength{\voffset}{4.2cm}

% command for blocks
\newcommand{\step}[6]{
\begin{minipage}{7.5cm}
{\tiny \color{#1}\texttt{#2} $\Rightarrow$ \texttt{#3}}\\
\fcolorbox{#1}{white}{
    \begin{minipage}{7.0cm}
    \begin{scriptsize}
    \texttt{#4} \hfill \color{gray}in: #5\color{black}\\
    \hphantom{m}\begin{minipage}[t]{6.8cm}#6\end{minipage}
    \end{scriptsize}
    \end{minipage}}
\end{minipage}
}

% command for the arrow with text
\newcommand{\longArrow}[1]{
\begin{minipage}[b]{7.5cm}
\fcolorbox{white}{white}{
\begin{minipage}[b]{7.0cm}
    \begin{center}
    \begin{scriptsize}
    $\overrightarrow{ %an arrow under which things may be written
      \begin{array}{c} % in order to be able to write multiple lines under the arrow
      #1\\
      \hphantom{\hspace{6.5cm}}
      \end{array}
    }$
    \end{scriptsize}
    \end{center}
\end{minipage}%
}
\end{minipage}%
\hphantom{ $\overrightarrow{}$}%
}

% command for the arrow between steps
\newcommand{\shortArrow}{$\overrightarrow{}$}

% command for marking things as model specific
\newcommand{\modelSpecific}{\emph{model specific}\xspace}

% the distance between two lines
\newcommand{\dummyDistance}{\\[4\baselineskip]}

% THE FLOW CHART STARTS HERE
\noindent
\step{black}{main()}{Dumux::start<ProblemTypeTag>() $\Rightarrow$ start\_()}{start\_()}{start.hh}%
     {start the simulation}
\shortArrow
\step{black}{start\_()}{timeManager.init()}{init()}{timemanager.hh}%
     {initialization}
\shortArrow
\step{black}{start\_()}{timeManager.run()}{run()}{timemanager.hh}%
     {time step management}
\dummyDistance
%
\longArrow{
    \textnormal{\texttt{while(!finished)}}\\
    \textnormal{\color{black}main}
    \rightarrow \textnormal{\color{dumuxBlue}time step}
    }
\step{dumuxBlue}{run()}{problem->timeIntegration()}{timeIntegration()}{implicitproblem.hh}%
     {execute time integration scheme}%
\longArrow{
    \textnormal{define number of allowed \textsc{Newton} fails}\\
    \textnormal{(each halving dt)}
    }
\dummyDistance
%
\step{dumuxBlue}{timeIntegration()}{model->update()}{update()}{implicitmodel.hh}%
     {sth like numerical model}
\shortArrow
\step{dumuxBlue}{update()}{solver.execute()}{execute()}{newtonmethod.hh}%
     {applying \textsc{Newton} method\\
      keeps track of things, catching errors}
\longArrow{
    \textnormal{\color{dumuxBlue}time step}
    \rightarrow \textnormal{\color{Mulberry}Newton step}\\
    \texttt{while(ctl.newtonProceed()}\\
    \textnormal{uLastIter = uCurrentIter(model.uCur())}
    }
\dummyDistance
%
\noindent
\step{Mulberry}{execute() $\Rightarrow$ execute\_()}{jacobianAsm.assemble()}{assemble()}{implicitassembler.hh}%
     {linearize the problem:\\
      add all element contributions to global \textsc{Jacobian}
      and global residual}%
\shortArrow
\step{Mulberry}{assemble() $\Rightarrow$ asImp\_().assemble\_()}{resetSystem\_()}{resetSystem\_()}{implicitassembler.hh}%
     {set r.h.s. (i.e. residual)\\
      set \textsc{Jacobian}  to zero }
\longArrow{
    \textnormal{\color{Mulberry}Newton step}
    \rightarrow \textnormal{\color{dumuxYellow}element}\\
    \texttt{loop all elements}\\
    }
\dummyDistance
%
\noindent
\step{dumuxYellow}{assemble() $\Rightarrow$ asImp\_().assemble\_()}{asImp\_().assembleElement\_()}{assembleElement\_()}{e.g. boxassembler.hh}%
     {call local \textsc{Jacobian} and residual assembly}%
\shortArrow
\step{dumuxYellow}{assembleElement\_()}{model\_().localJacobian().assemble()}{assemble()}{implicitlocaljacobian.hh}%
     {set curr. element, update element's fin.vol.geom.\\
      reset local \textsc{Jacobian} to 0\\
      update types of boundaries on this element}%
\shortArrow
\step{dumuxYellow}{assemble()}{prevVolVars\_.update(),curVolVars\_.update()}{update()}{e.g. 2pvolumevariables.hh}%
     {call model (e.g. \texttt{2p})specific update of quantities defined for the volume:\\
      variables for the \emph{current} and \emph{previous} timestep}%
\dummyDistance
%
\noindent
\step{dumuxYellow}{update()}{completeFluidState()}{completeFluidState()}{e.g. 2pvolumevariables.hh}%
     {calculate all required fluid properties from the primary variables,
      here the fluid system does the real work:\\
      calculates, saves, and provides: densities, etc.}
\shortArrow
\step{dumuxYellow}{assemble()}{localResidual().eval()$\Rightarrow$asImp\_().eval()}{eval()}{e.g. implicitlocalresidual.hh}%
     {the element's local residual is calculated:\\
      see the next two stepstones}%
\shortArrow
\step{dumuxYellow}{eval()}{asImp\_().evalFluxes\_()}{evalFluxes\_()}{e.g. boxlocalresidual.hh}%
     {evaluate the fluxes going into each finite volume,
      this is \modelSpecific}
\dummyDistance
%
\step{dumuxYellow}{evalFluxes\_()}{this$\rightarrow$asImp\_().computeFlux()}{computeFlux()}{e.g. 2plocalresidual.hh}%
     {this calculate the \modelSpecific fluxes (e.g. advective and diffusive)
      using the \texttt{FluxVariables}}
\shortArrow
\step{dumuxYellow}{eval()}{asImp\_().evalVolumeTerms\_()}{evalVolumeTerms\_()}{implicitlocalresidual.hh}%
     {evaluate the \modelSpecific storage and source terms for each finite volume}%
\shortArrow
\step{dumuxYellow}{eval()}{asImp\_().evalBoundary\_()}{evalBoundary\_()}{implicitlocalresidual.hh}%
     {evaluate the \modelSpecific boundary conditions}%
\dummyDistance
%
\step{dumuxYellow}{assemble()}{asImp\_().evalPartialDerivative\_()}{evalPartialDerivative\_()}{e.g. implicitlocaljacobian.hh}%
     {actually calculate the element's (local) \textsc{Jacobian}\\
      matrix a property chooses backward/central/foward\\
      differences. here: central differences}
\shortArrow
\begin{minipage}{0.50\textwidth}
  \begin{scriptsize}\textnormal{approximation of partial derivatives: numerical differentiation}\end{scriptsize}\\
  \begin{scriptsize}\textnormal{add $\pm \epsilon$ solution, divide difference of residual by $2\epsilon$}\end{scriptsize}\\
  \begin{scriptsize}\textnormal{all partial derivatives for the element from the local \textsc{Jacobian} matrix}\end{scriptsize}\\
  $\left \lbrace
      \begin{tabular}{l}%these question marks are for the \verb, not meant as ``unclear''
          \verb?priVars[pvIdx]+=eps?\\
          \begin{scriptsize}\textnormal{this is adding eps to the current solution}\end{scriptsize}\\
      \verb?curVolVars_[scvIdx].update(+eps)?\\
          \begin{scriptsize}\textnormal{recalculate volume variables, having $\epsilon$ added}\end{scriptsize}\\
          \verb?localResidual().eval(+eps)?\\
          \begin{scriptsize}\textnormal{calculate local residual for modified solution as before: involves}\end{scriptsize}\\
      {\scriptsize $\begin{array}{l}
          \textnormal{- \textbf{computeFlux}}\\
          \textnormal{- \textbf{computeStorage}}\\
          \textnormal{- \textbf{computeSource}} \\
      \end{array}$} \\
      \verb?store the residual()?\\
      \verb?repeat for priVars[pvIdx]-=eps?\\
      \verb?derivative is (residual(+eps) - residual(-eps))/2eps?\\
    \end{tabular}
    \right .
  $
\end{minipage}
\dummyDistance
%
\step{dumuxYellow}{assemble\_()}{asImp\_().assembleElement\_()}{assembleElement\_()}{implicitassembler.hh}%
     {Residual of the current solution is now\\
      ``numerically differentiated'', for the element i.e.\\
      the local \textsc{Jacobian} matrix is calculated. }%
\longArrow{
    \textnormal{The contribution of a single element is done.}\\
    \textnormal{Now, it needs to be added to the global quantities:}\\
    \textnormal{Add to global residual and global \textsc{Jacobian}}.}
\step{dumuxYellow}{assemble\_()}{asImp\_().assembleElement\_()}{assembleElement\_()}{e.g. boxassembler.hh}
     {Add to global residual.:\\
      \texttt{resdidual\_[globI+=\\model\_().globalJacobian().resdidual(i)]}}
\dummyDistance
%
\longArrow{
    \textnormal{loop vertices}\\
    \textnormal{of an element}
    }
\step{dumuxYellow}{assemble\_()}{asImp\_().assembleElement\_()}{assembleElement\_()}{e.g. boxassembler.hh}
     {Add to global residual:\\
      \texttt{(*matrix\_)[globI][globJ] +=\\model\_().localJacobian().mat(i,j)}}
\longArrow{
    \textbf{\textbf{\color{dumuxYellow}element}}
    \rightarrow \textbf{\color{Mulberry}Newton step}\\
    \textnormal{Assembling of elements to global quantities is done.}
    }
\dummyDistance
%
\step{Mulberry}{execute\_()}{while(ctl.newtonProceed())}{newtonProceed()}{newtoncontroller.hh}%
     {Print information.\\
      Start/ stop timer.}%

\longArrow{
    \textnormal{set delta Vector to zero} \\
    \textnormal{(this is what is}\\
    \textnormal{solved for later)}\\
    }
\step{Mulberry}{execute\_()}{ctl.newtonSolveLinear()}{newtonSolveLinear()}{newtoncontroller.hh}%
     {Catching errors.\\
      Ask the linear solver to solve the system.\\
      i.e.: give \textsc{Jacobian}(matrix), delta(x), r.h.s.(residual) to linear solver\\
      $\nabla r(x^k) \cdot \Delta x^k = r(x^k)$\\
      tricky: each \textsc{Newton} step solves a linear system of equations.}%
\shortArrow
\step{Mulberry}{newtonSolveLinear()}{int converged = linearSolver\_.solve()}{solve()}{boxlinearsolver.hh}%
     {Solve the linear system with the chosen backend.}%
\dummyDistance
%
\step{Mulberry}{execute\_()}{ctl.newtonUpdate()}{newtonUpdate()}{newtoncontroller.hh}%
     {We solved for the change in solution, but need the solution:\\
      Calculate current (this iteration) solution\\
      \quad from last (iteration) solution and current (iteration) change in  solution:\\
      $x^{k+1} = x^k - \Delta x^k$ where $\Delta x^k = (\nabla r(x^k))^{-1} \cdot r(x^k)$}
\shortArrow
\step{Mulberry}{execute\_()}{ctl.newtonEndStep()}{newtonEndStep()}{newtoncontroller.hh}%
     {Increase counter for number of \textsc{Newton} steps.\\
      Print info.}%
\longArrow{
    \textnormal{check whether to do another \textsc{Newton} iteration:} \\
    \textnormal{that is: check if the error is below tolerance or}\\
    \textnormal{maximum number of iterations was reached.}
    }
\dummyDistance
%
\longArrow{
    \textbf{\textbf{\color{Mulberry}Newton step}}
    \rightarrow \textbf{\color{dumuxBlue}Time step}\\
    \textnormal{\textsc{Newton} done}\\
    \textnormal{if failed $\rightsquigarrow$ halve timestep size, restart loop}\\
    }
\step{dumuxBlue}{execute\_()}{ctl.newtonEnd()}{newtonEnd()}{newtoncontroller.hh}%
     {Tell the controller we are done}%
\shortArrow
\step{dumuxBlue}{update()}{asImp\_().updateSuccessful()}{updateSuccessful()}{e.g. implicitmodel.hh}%
     {can be filled \modelSpecific}%
\dummyDistance
%
\longArrow{
    \textnormal{in while(!finished)}
    }
\step{dumuxBlue}{run()}{problem\_->postTimeStep()}{postTimeStep(),writeOutput()}{implicitproblem.hh}%
     {Give the problem the chance to post-process the solution.}%
\longArrow{
    \textnormal{write output}\\
    \textnormal{uPrev $\leftarrow$ uCur}\\
    \textnormal{time += dt, timestepIdx++}\\
    \textnormal{deal with restart and episodes }
    }
\dummyDistance
%
\step{dumuxBlue}{run()$\Rightarrow$setTimeStepSize(problem\_->nextTimeStepSize(dt))\\
                 $\Rightarrow$nextTimeStepSize()}
     {newtonCtl\_.suggestTimestepSize()}{suggestTimestepSize()}{newtoncontroller.hh}%
     {Determine new time step size from number of \textsc{Newton} steps.}%
\longArrow{
  \textbf{\color{dumuxBlue}Time step}
  \rightarrow \textbf{\color{black}main}\\
  \textnormal{loop until simulation is finished}
}

\end{landscape}

\newpage
% Original pagestyle (headings and footer) were switched off,
% in order to get more space for the flowchart.
\pagestyle{scrheadings}
\normalsize
