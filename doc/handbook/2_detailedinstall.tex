\section{Detailed Installation Instructions}
\label{install}

In this section about the installation of \Dumux it is assumed that you work with
a Linux or Apple OS X operating system
and that you are familiar with the use of a command line shell. Installation means
that you unpack \Dune together with \Dumux in a certain directory.
Then, you compile it in that directory tree in which you do the further work, too.
You also should know how to install new software packages
or you should have a person on hand who can give you assistance with that. In
section \ref{sec:prerequisites} we list some prerequisites for running \Dune and \Dumux.
Please check in said paragraph whether you can fulfill them. In addition, section
\ref{sec:external-modules-libraries} provides some details on optional libraries and modules.

In a technical sense \Dumux is a module of \Dune.
Thus, the installation procedure of \Dumux is the same as that of \Dune.
Details regarding the installation of \Dune are provided on the \Dune website \cite{DUNE-HP}.
If you are interested in more details about the build system that is used,
they can be found in the \Dune buildsystem documentation\footnote{\url{https://www.dune-project.org/buildsystem/}} and
CMake's documentation\footnote{\url{https://cmake.org/documentation/}}.

All \Dune modules, including \Dumux, get extracted into a common directory, as it
is done in an ordinary \Dune installation.
We refer to that directory abstractly as {\Dune} root directory or, in short, as {\Dune}-Root.
If it is used as directory's path of a shell command it is typed as \texttt{\Dune-Root}.
For the real {\Dune} root directory on your file system any valid directory name can be chosen.

Source code files for each \Dune module are contained in their own subdirectory
within {\Dune}-Root.
We name this directory of a certain module \emph{module root directory} or
\texttt{module-root-directory} if it is a directory path,
e.\,g. for the module \texttt{dumux} these names are  \emph{dumux root directory}
respective \texttt{dumux-root-directory}.
The real directory names for the modules can be chosen arbitrarily. In this manual
they are the same as the
module name or the module name extended by a version number suffix.
The name of each \Dune module is defined in the file \texttt{dune.module}, which is
in the root
directory of the respective module. This should not be changed by the user.

After extracting the source code for all relevant \Dune modules, including \Dumux, \Dune has to be built
by the shell-command \texttt{dunecontrol} which is part of the \Dune build system.

\subsection{Prerequisites} \label{sec:prerequisites}
A reasonable recent \Cplusplus compiler (g++ (4.9), clang++ (3.5), or Intels ICC), CMake (version 2.8.12 or newer) and their
dependencies are required.
For prerequisite software packages to install see \cite{DUNE-HP}.

The building of included documentation like this handbook requires \LaTeX{} and auxiliary tools
\texttt{bibtex}. One usually chooses a \LaTeX{} distribution like \texttt{texlive} for this purpose.
It is possible to switch off the building of the documentation by setting the switch \texttt{--disable-documentation}
in the \texttt{CONFIGURE\_FLAGS} of the building options, see Chapter \ref{buildIt}.
Additional parts of documentation are contained within the source code files as special formatted comments.
Extracting them can be done using \texttt{doxygen}, cf. Section \ref{sec:build-doxy-doc}.

Depending on whether you are going to use external libraries and modules for additional \Dune features,
additional software packages may be required. Some hints on that are given in Section \ref{sec:external-modules-libraries}.

Git clients must be installed to download modules from Git repositories.

\subsection{Obtaining Source Code for \Dune and \Dumux}
As stated above, the \Dumux release and trunk (developer tree) are based on the most recent
\Dune release 2.4, comprising the core modules dune-common, dune-geometry, dune-grid,
dune-istl and dune-localfunctions. For working with \Dumux, these modules are required. The
external module dune-PDELab is recommended and required for several \Dumux features.

Two possibilities exist to get the source code of \Dune and \Dumux.
Firstly, \Dune and \Dumux can be downloaded as tar files from the respective \Dune and \Dumux website.
They have to be extracted as described in the next paragraph.
Secondly, a method to obtain the most recent source code (or, more generally, any of its previous revisions) by direct access
to the software repositories of the revision control system is described in the subsequent part.

However, if a user does not want to use the most recent version,
certain version tags or branches (i.\,e. special names) are means
of the software revision control system to provide access to different versions of the software.

\paragraph{Obtaining the software by installing tar files}
The slightly old-fashionedly named tape-archive-file, shortly named tar file or
tarball, is a common file format for distributing collections of files contained
within these archives.
The extraction from the tar files is done as follows:
Download the tarballs from the respective \Dune (version 2.4.1) and \Dumux websites
to a certain folder in your file system.
Create the {\Dune} root directory, named \texttt{dune} in the example below.
Then extract the content of the tar files, e.\,g. with the command-line program
\texttt{tar}.
This can be achieved by the following shell commands. Replace \texttt{path\_to\_tarball}
with the directory name where the downloaded files are actually located.
After extraction, the actual name of the \emph{dumux root directory} is \texttt{dumux-\DumuxVersion}
(or whatever version you downloaded).

\begin{lstlisting}[style=Bash]
$ mkdir dune
$ cd dune
$ tar xzvf path_to_tarball_of/dune-common-2.4.1.tar.gz
$ tar xzvf path_to_tarball_of/dune-geometry-2.4.1.tar.gz
$ tar xzvf path_to_tarball_of/dune-grid-2.4.1.tar.gz
$ tar xzvf path_to_tarball_of/dune-istl-2.4.1.tar.gz
$ tar xzvf path_to_tarball_of/dune-localfunctions-2.4.1.tar.gz
$ tar xzvf path_to_tarball_of/dune-pdelab-2.0.0.tar.gz
$ tar xzvf path_to_tarball_of/dune-typetree-2.4.1.tar.gz
$ tar xzvf path_to_tarball_of/dumux-2.9.tar.gz
\end{lstlisting}

Furthermore, if you wish to install the optional \Dune Grid-Howto which provides a tutorial
on the Dune grid interface, act similar.

\paragraph{Obtaining \Dune and \Dumux from software repositories}
Direct access to a software revision control system for downloading code can be of advantage later on.
It is easier to keep up with code changes and to receive important bug fixes.\Dune and \Dumux use
Git for their software repositories. To access them a Git client is needed.

In the technical language of Git, \emph{cloning a certain software version} means nothing more then fetching
a local copy from the software repository and laying it out in the file system.
In addition to the software, some more files for the use of the software revision
control system itself are created. If you have developer access to \Dumux, it is
also possible to do the opposite, i.\,e. to load up a modified revision of software
into the software repository. This is usually termed as \emph{commit} and \emph{push}.

The installation procedure is done as follows:
Create a  {\Dune} root directory, named e.g. \texttt{DUNE-ROOT} in the lines below.
Then, enter the previously created directory and check out the desired modules.
As you see below, the check-out uses two different servers for getting the sources,
one for \Dune and one for \Dumux.
The \Dune modules of the stable 2.4 release branch are checked out as described
on the \Dune website \cite{DUNE-HP}:

\begin{lstlisting}[style=Bash]
$ mkdir DUNE-ROOT
$ cd DUNE-ROOT
$ git clone -b releases/2.4 https://gitlab.dune-project.org/core/dune-common.git
$ git clone -b releases/2.4 https://gitlab.dune-project.org/core/dune-geometry.git
$ git clone -b releases/2.4 https://gitlab.dune-project.org/core/dune-grid.git
$ git clone -b releases/2.4 https://gitlab.dune-project.org/core/dune-istl.git
$ git clone -b releases/2.4 https://gitlab.dune-project.org/core/dune-localfunctions.git
$ git clone -b releases/2.3 https://gitlab.dune-project.org/PDELab/dune-typetree.git
$ git clone -b releases/2.0 https://gitlab.dune-project.org/PDELab/dune-pdelab.git
\end{lstlisting}

The newest and maybe unstable developments are also provided in these repositories and can be found in the \emph{master} branch.
Please check the \Dune website \cite{DUNE-HP} for further information. We always try to keep up with the latest developments of \Dune.
However, the current \Dumux release is based on the stable 2.4 release and it might not compile without further adaptations using the newest versions of \Dune.

Furthermore, if you wish to install the optional \Dune Grid-Howto which provides a tutorial
on the Dune grid interface, act similar.

The \texttt{dumux} module is checked out as described below (see also the \Dumux website:
\url{http://www.dumux.org/}).
Its file tree has to be created in the \Dune-Root directory, where the \Dune modules
have also been checked out to. Subsequently, the next command
is executed there, too. The dumux root directory is called \texttt{dumux} here.

\begin{lstlisting}[style=Bash]
$ # make sure you are in DUNE-Root
$ git clone https://git.iws.uni-stuttgart.de/dumux-repositories/dumux.git
\end{lstlisting}

\paragraph{Patching \Dune or external libraries}
\label{sc:patchingDUNE}
Patching of \Dune modules in order to work together with \Dumux can be necessary for several reasons.
Software like a compiler or even a standard library
changes at times. But, for example, a certain release of a software component that we depend on,
may not reflect that change and thus it has to be modified.
In the dynamic developing process of software which depends on other modules it is not always feasible
to adapt everything to the most recent version of each module. They may fix problems with a certain module
of a certain release without introducing too much structural change.

\Dumux contains patches and documentation about their usage and application within the
directory \texttt{dumux/patches}.
Please check the README file in that directory for recent information.
In general, a patch can be applied as follows
(the exact command or the used parameters may be slightly different).
We include here an example of a patching dune-grid.

\begin{lstlisting}[style=Bash]
$ # make sure you are in DUNE-Root
$ cd dune-grid
$ patch -p0 < ../dumux/patches/grid-2.3.1.patch
\end{lstlisting}

It can be removed by
\begin{lstlisting}[style=Bash]
$ path -p0 -R < ../dumux/patches/grid-2.3.1.patch
\end{lstlisting}

The \texttt{checkout-dumux} script also applies patches, if not explicitly requested not to do so.

\paragraph{Hints for \Dumux-Developers}
If you also want to actively participate in the development of \Dumux, you can allways send patches
to the Mailing list.

To get more involved, you can apply either for full developer
access or for developer access on certain parts of \Dumux. Granted developer access means that
you are allowed to commit own code and that you can access the \texttt{dumux-devel} module.
This enhances \texttt{dumux} by providing maybe unstable code from the developer group.

\subsection{Building Documentation}
\subsubsection{Doxygen}
\label{sec:build-doxy-doc}
Doxygen documentation is done by especially formatted comments integrated in the source code,
which can get extracted by the program \texttt{doxygen}. Beside extracting these comments,
\texttt{doxygen} builds up a web-browsable code structure documentation
like class hierarchy of code displayed as graphs, see \url{http://www.stack.nl/~dimitri/doxygen/}.

The Doxygen documentation of a module can be built, if \texttt{doxygen} is installed,
by running \texttt{dunecontrol}, entering the \texttt{build-*}directory, and execute
\texttt{make doc}. Then point your web browser to the file
\texttt{MODULE\_BUILD\_DIRECTORY/doc/doxygen/html/index.html} to read the generated documentation.
This should also work for other \Dune modules.

\subsubsection{Handbook}
To build the \Dumux handbook go into the \texttt{build-}directory and
run \texttt{make doc} or \texttt{make 0\_dumux-handbook\_pdf}. The pdf can then be found
in \texttt{MODULE\_BUILD\_DIRECTORY/doc/handbook/0\_dumux-handbook.pdf}.

\subsection{External Libraries and Modules} \label{sec:external-modules-libraries}
The libraries described below provide additional functionality but are not generally required to run \Dumux.
If you are going to use an external library check the information provided on the \Dune website%
\footnote{DUNE: External libraries, \url{https://www.dune-project.org/doc/external-libraries/}}.
If you are going to use an external \Dune module the website on external modules%
\footnote{DUNE: External modules, \url{https://www.dune-project.org/groups/external/}}
can be helpful.

Installing an external library can require additional libraries which are also used by \Dune.
For some libraries, such as BLAS or MPI, multiple versions can be installed on the system.
Make sure that it uses the same library as \Dune when configuring the external library.

Some of the libraries are then compiled within that directory and are not installed in
a different place, but \Dune may need to know their location. Thus, one may have to refer to
them as options for \texttt{dunecontrol}, for example via the options file \texttt{my-debug.opts}.
Make sure you compile the required external libraries before you run \texttt{dunecontrol}.

An easy way to install some of the libraries and modules given below is the
\texttt{installexternal.sh} script located in \texttt{bin}. The script
has to be called from your {\Dune} root directory.


\subsubsection{List of External Libraries and Modules}
In the following list, you can find some external modules and external libraries,
and some more libraries and tools which are prerequisites for their use.

\begin{itemize}
\item \textbf{dune-ALUGrid}: Grid library, comes as a \Dune module.
  The parallel version needs also a graph partitioner, such as {ParMETIS}.
  Download: \url{https://gitlab.dune-project.org/extensions/dune-alugrid}

\item \textbf{dune-foamgrid}: External grid module. One- and two-dimensional grids
  in a physical space of arbitrary dimension; non-manifold grids, growth, element
  paramterizations, and movable vertices. This makes FoamGrid the grid data structure
  of choice for simulating structures such as foams, discrete fracture networks,
  or network flow problems.
  Download: \url{https://gitlab.dune-project.org/extensions/dune-foamgrid}

\item \textbf{\Dune-multidomaingrid} and \textbf{\Dune-multidomain}: External modules which offer a meta grid that
  has different sub-domains. Each sub-domain can have a local operator that is coupled by a coupling condition. They are
  used for multi-physics approaches or domain decomposition methods. Download:
  \url{https://github.com/smuething/dune-multidomaingrid}
  and \url{https://github.com/smuething/dune-multidomain}

\item \textbf{\Dune-PDELab}: External module to write more easily discretizations. PDELab provides
  a sound number of discretizations like FEM or discontinuous Galerkin methods.
  Download: \url{https://gitlab.dune-project.org/pdelab/dune-pdelab}

\item \textbf{PARDISO}: External library for solving linear equations. The package PARDISO is a thread-safe,
  high-performance, robust, memory efficient and easy to use software for solving large sparse symmetric
  and asymmetric linear systems of equations on shared memory multiprocessors. The precompiled binary
  can be downloaded after personal registration from the PARDISO website: \url{http://www.pardiso-project.org}

\item \textbf{SuperLU}: External library for solving linear equations. SuperLU is a general purpose
  library for the direct solution of large, sparse, non-symmetric systems of linear equations.
  Download: \url{http://crd.lbl.gov/~xiaoye/SuperLU}

\item \textbf{UMFPack}: External library for solving linear equeations. It is part of SuiteSparse.

\item \textbf{UG}: External library for use as grid. UG is a toolbox for unstructured grids, released under GPL.
  To build UG the tools \texttt{lex}/\texttt{yacc} or the GNU variants of \texttt{flex}/\texttt{bison} must be provided.
  Download: \url{https://gitlab.dune-project.org/staging/dune-uggrid}
\end{itemize}

The following are dependencies of some of the used libraries. You will need them
depending on which modules of \Dune and which external libraries you use.

\begin{itemize}
\item \textbf{MPI}: The parallel version of \Dune and also some of the external dependencies need MPI
  when they are going to be built for parallel computing. \texttt{OpenMPI} and \texttt{MPICH} in a recent
  version have been reported to work.

\item \textbf{BLAS}: SuperLU makes use of BLAS. Thus install GotoBLAS2, ATLAS, non-optimized BLAS
  or BLAS provided by a chip manufacturer. Take care that the installation scripts select the intended
  version of BLAS.

\item \textbf{METIS} and \textbf{ParMETIS}: This are dependencies of ALUGrid and can be used with UG, if run in parallel.

\item \textbf{Compilers}: Beside \texttt{g++}, \Dune can be built with Clang from the LLVM project and
  Intel \Cplusplus compiler. C and Fortran compilers are needed for some external libraries. As code of
  different compilers is linked together they have to be be compatible with each other.
\end{itemize}
