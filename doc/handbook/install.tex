\section{Installation} 
\label{install}

For the installation of DuMu$^\text{x}$, the following steps have to be performed.  

\paragraph{Checkout of the core modules}
Since we always want to be up to date with the latest changes in DUNE, 
we decided to use the developers version of the core modules 
\texttt{dune-common}, \texttt{dune-grid}, \texttt{dune-istl}, and \texttt{dune-disc}.  
First, create a directory where all the DUNE modules will be stored in. Then, enter the previously created folder. 
The checkout has to be performed as described on 
the DUNE webpage, \cite{DUNE-HP}: 
\begin{itemize}
\item \texttt{svn checkout https://svn.dune-project.org/svn/dune-common/trunk dune-common}
\item \texttt{svn checkout https://svn.dune-project.org/svn/dune-grid/trunk dune-grid}
\item \texttt{svn checkout https://svn.dune-project.org/svn/dune-istl/trunk dune-istl}
\item \texttt{svn checkout https://svn.dune-project.org/svn/dune-disc/trunk dune-disc}
\end{itemize} 
The \Dumux webpage \cite{dumux-hp} contains the revision numbers of the core modules 
for which compatibility to \Dumux can be guaranteed. 

Additionally, it is highly recommended that you also checkout the \texttt{dune-grid} HOWTO 
by 
\begin{center}
\texttt{svn checkout https://svn.dune-project.org/svn/dune-grid-howto/trunk dune-grid-howto}
\end{center}
and work yourself through that tutorial in order to get an understanding of 
the Dune grid interface. 

\paragraph{Checkout of DuMu$^\text{x}$ and external modules} 
If you obtained a \Dumux tarball, you should just extract the tarball as usual 
and can skip this part. 
First of all, you need to ask one of the IWS system administrators to 
add your account to the group \texttt{svndune}. 
If you are working on a LH2 computer, you then can checkout DuMu$^\text{x}$ 
and the external modules via 
\begin{itemize}
\item \texttt{svn checkout svn+ssh://luftig/home/svn/DUMUX/dune-mux/trunk dumux}
\item \texttt{svn checkout svn+ssh://luftig/home/svn/DUMUX/dune-subgrid/trunk dune-subgrid}
\item \texttt{svn checkout svn+ssh://luftig/home/svn/DUMUX/external/trunk external}
\end{itemize} 
If you want to checkout from outside LH2, you first need to establish a tunnel. 
To this end, you need to add once 
\begin{center}
\texttt{ssht = ssh -p 2022 -l login -o HostKeyAlias=luftig.iws.uni-stuttgart.de} 
\end{center}
with \texttt{login} replaced 
by your actual IWS login name, to the section \texttt{[tunnels]} of the 
file \texttt{\$HOME/} \texttt{.subversion/config}, 
The tunnel then needs to be initialized every time you want 
to connect to the repository by 
\begin{center}
\texttt{ssh -Nf -L 2022:luftig.iws.uni-stuttgart.de:22 login@login1.iws.uni-stuttgart.de}.
\end{center}
Then, you can checkout everything as described above, if you replace \texttt{luftig} 
by \texttt{localhost}. 

%Moreover, the possibility to checkout a read-only version from outside exists. Therefor, no IWS-account is needed. The following commands can be used for an anonymous checkout:
%\begin{itemize}
%\item \texttt{svn checkout svn://svn.iws.uni-stuttgart.de/DUMUX/dune-mux/trunk dumux}
%\item \texttt{svn checkout svn://svn.iws.uni-stuttgart.de/DUMUX/external/trunk external}
%\item \texttt{svn checkout svn://svn.iws.uni-stuttgart.de/DUMUX/dune-subgrid/trunk dune-subgrid}
%\end{itemize} 

\paragraph{Build the external modules} 
If you obtained a \Dumux tarball, you should skip this part. 
The external modules have to be built first. They consist of Alberta, ALUGrid, UG, and METIS.
To install them all, execute the install script in the folder \texttt{external}:
\begin{center}
\texttt{./installExternal.sh all}
\end{center}
If you like to only install some of the external software, you can choose one of the 
corresponding options instead of \texttt{all}: \texttt{alberta, alu, metis, ug}.
Please also refer to the DUNE webpage for additional details, \cite{DUNE-HP}. 

\paragraph{Build DUNE and DuMu$^\text{x}$}
Type in the folder \texttt{DUMUX}: 
\begin{center}
\texttt{./dune-common/bin/dunecontrol --opts=\$DUMUX\_ROOT/debug.opts all}
\end{center}
This uses the DUNE buildsystem. If it does not work, please consider
the file \texttt{INSTALL} in the \Dumux root directory (if you use
SVN, this \texttt{\$DUMUX\_ROOT} is usually \texttt{dumux}, if you use
a released version it is usually \texttt{dumux-VERSION}). You can also
find more information in the DUNE Buildsystem HOWTO located at the
DUNE webpage, \cite{DUNE-HP}.  Alternatively, the tool CMake can be
used to build \Dumux. Please check the file \texttt{INSTALL.cmake} for
details.
