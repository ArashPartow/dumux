So, you've downloaded your very own copy of \Dumux and its dependencies.
You've run dunecontrol, and your example ``test$\_$dumux" not only complies,
but it even shows a nice simulation in paraview. 
Maybe you've read through parts of the handbook, and even started looking 
though the doxygen documentation. 
Well done. What now? \\ \linebreak

\textit{``How on earth is this going to help me solve my multi-(phase, component, 
scale, physics) flow and transport problems in porous media systems?''}, you begin to wonder.
Don't panic! In order to best ease our prospective users and developers into the
wonderful \Dumux simulation environment, we've prepared a \Dumux-course. 
This course is offered once a year over a period of 3 days at the University of Stuttgart.
If you're looking for information on attending, subscribe to the \Dumux mailing list 
and stay tuned for updates. \\
\url{https://listserv.uni-stuttgart.de/mailman/listinfo/dumux} \\ \linebreak

\textit{``But the course won't take place for another 6 months!"} and, 
\textit{``I want to start developing a numerical model of my challenging and 
	interesting process now!"}, you think. 
Not a problem. The course materials are all shared online in their own 
\Dumux project repository. A series of beginner-level exercises are explained 
such that you can see how a model is developed in \Dumux. As a teaser, we've
 also included a suite of examples from hot topics we're working on. Models
  exploring ``Coupling free flow and porous-media flow", ``Flow in fractured
   porous media" and ``Fluid-solid phase change" are all introduced.  \\ \linebreak
   
\textit{``Sounds great, but where is this material? I can't find it within
what I've downloaded."}, you question. 
The dumux course material is available online: \\
\url{https://git.iws.uni-stuttgart.de/dumux-repositories/dumux-course} \\
In order to download this repository, which acts as an additional module to 
the \Dumux base, you can download an installation script with the following command:
\begin{lstlisting}[style=Bash]
$ wget https://git.iws.uni-stuttgart.de/dumux-repositories/dumux-course/raw/master/scripts/install.sh
\end{lstlisting}
This script will install \Dumux, it's Dune dependencies, and the \Dumux-course 
repository. Within the directory \Dumux-course there are a series of exercises 
and slides describing the previously described examples. \\ \linebreak

The \Dumux-course is dependent on a certain \Dumux release. The above script will
download the correct version (\textbf{releases\textbackslash3.0}). You can check which version
of \Dumux you have installed using the git status command. In the future, the \Dumux-course
will be updated to depend on the specific release, (\textbf{releases\textbackslash3.0}, \textbf{releases\textbackslash3.0}),
that was closest to the date of the course.