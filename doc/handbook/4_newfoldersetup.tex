\section{Setup of a New Folder and New Tests}

\paragraph{Setting up a New Folder}
In this section it is described how to set up a new folder and how to tell
the build system, that there is a new one.

\begin{enumerate}[1)]
 \item create new folder with content
 \item adapt the \verb+CMakeList.txt+ in the folder above and add a line with
       \verb+add_subdirectory(NEW\_FOLDER)+
 \item adapt the \verb+CMakeList.txt+ in the newly created folder and add your test
       (see below for more information)
 \item go to your \texttt{build}-directory and type \verb+make+ to
       reconfigure the system
\end{enumerate}

\paragraph{Adding a New Test Program}
\noindent To simply add a new executable use the following macro. The test will \emph{not} be built
automatically when running \texttt{ctest}. You have to compile it manually by
\texttt{make test\_program}.
\begin{verbatim}
add_executable_all(test\_program test\_program.cc)
\end{verbatim}

\noindent To add a test, which should be compiled when running \texttt{ctest}, use the
\texttt{add\_dumux\_test} macro. You can decide whether, the program should be run
after compiling or not.
Please note that the name of the test (first argument) must be unique, whereas the name
of the executable (second argument) can occur multiple times.
\begin{verbatim}
add_dumux_test(test\_program test\_program test\_program.cc
  test\_program # add this line, if the program should also be run
  )
\end{verbatim}

\noindent To add a test which should be run and compared to a reference solution when using
\texttt{ctest}, please use the following structure. The macro \texttt{\${CMAKE\_SOURCE\_DIR}}
gives the location of your source code. The macro \texttt{\${CMAKE\_CURRENT\_BINARY\_DIR}}
gives the current folder with the executable.
\begin{verbatim}
add_dumux_test(test\_program test\_program test\_program.cc
  ${CMAKE_SOURCE_DIR}/bin/runTest.sh
  ${CMAKE_SOURCE_DIR}/bin/fuzzycomparevtu.py
  LOCATION_TO_THE_REFERNCE_SOLUTION/test\_program-reference.vtu
  ${CMAKE_CURRENT_BINARY_DIR}/test\_program-00009.vtu
  ${CMAKE_CURRENT_BINARY_DIR}/test\_program)
\end{verbatim}

\paragraph{Committing a New older to SVN}
For those who work with Subversion (\texttt{svn}) and want to commit a newly setup folder to the repository some basics are
given in this paragraph. For further reading please check out the Subversion User Manual found at \cite{APACHE-SUBVERSION-HP}
where you will also find a "High Speed Turorial" in the appendix. \\
The four most important commands are \texttt{svn checkout}, \texttt{svn update},  \texttt{svn add}
and \texttt{svn commit}. The first one (\texttt{svn checkout}) you probably already know from the \Dumux installation.
It will create a copy of the trunk version from the svn server on your local system. Use \texttt{svn update} to get the
latest changes in the repository (commits from other users). In order to add a new folder to the repository the following
steps have to be taken:

\begin{enumerate}[1)]
\item \texttt{svn update}: The first step is to update your \Dumux. You should execute this command in your
      dumux-stable or dumux-devel folder.
\item \texttt{svn add --depth=empty YOURFOLDER}: This command adds the folder without its content.
\item In your folder: use \texttt{svn add YOURFILES} to add your files. Generally, you should only add
      your header files (.hh), your source files (.cc), your input file (.input), if required your
      grid file (.dgf) or if necessary other text-based files. Please do not upload (large) binary files.
\item Type \texttt{svn status} in your \texttt{dumux}-root directory the see all the file changes.
      \texttt{?} indicates possible forgotten files. Make sure that you include all necessary
      files in your commit.
\item Use \texttt{svn commit} from the directory level containing your folder. This uploads all your changes to the
      svn server. You will be asked to briefly explain the content of your commit in an editor.
\end{enumerate}
