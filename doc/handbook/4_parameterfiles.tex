\section{Parameters in \Dumux}
\label{sc_parameterfiles}
Simulation parameters can be parsed to the program via a parameter file or the command line.
A list of all available parameters is provided in the Doxygen documentation
of the file \texttt{parameterfile}, which is accessible via \texttt{Modules -> Parameters}.

After having run the example application from section \ref{quick-start-guide} you will
get the following output at the end of the simulation run
\footnote{If you did not get the output, restart the application the following way:
\texttt{./test{\_}box2p -PrintParameters true},
this will print the parameters once your simulation is finished}:
\begin{lstlisting}[style=Bash]
# Run-time specified parameters:
[ Grid ]
File = "./grids/test_2p.dgf"
[ Implicit ]
EnableJacobianRecycling = "1"
EnablePartialReassemble = "1"
[ Problem ]
Name = "lensbox"
[ SpatialParams ]
LensLowerLeftX = "1.0"
LensLowerLeftY = "2.0"
LensUpperRightX = "4.0"
LensUpperRightY = "3.0"
[ TimeManager ]
DtInitial = "250"
TEnd = "3000"
# DEPRECATED run-time specified parameters:
PrintParameters = "1"
# Replace by:
[ TimeManager ]
PrintParameters = "1"
# Compile-time specified parameters:
[ Implicit ]
EnableHints = "0"
MassUpwindWeight = "1"
MaxTimeStepDivisions = "10"
MobilityUpwindWeight = "1"
NumericDifferenceMethod = "1"
UseTwoPointFlux = "0"
[ LinearSolver ]
MaxIterations = "250"
PreconditionerRelaxation = "1"
ResidualReduction = "1e-06"
Verbosity = "0"
[ Newton ]
WriteConvergence = "0"
[ Problem ]
EnableGravity = "1"
[ TimeManager ]
MaxTimeStepSize = "1.79769e+308"
[ Vtk ]
AddVelocity = "0"
# UNUSED parameters:
ImportantVariable = "1"
\end{lstlisting}

A number of things can be learned:
\begin{itemize}
  \item \emph{run-time} parameters can be changed without re-compiling
  \item \emph{deprecated run-time} parameters will be removed in the next release
  \item \emph{compile-time} parameters cannot be overwritten by the input file
  \item \emph{unused} are not used by the simulation (maybe typo or wrong group)
\end{itemize}

All applications have a help message which you can read by giving
\texttt{--help} as a command line argument to the application.

For further details, please have a look for \texttt{Dune::ParameterTree}
in the \Dune documentation.
