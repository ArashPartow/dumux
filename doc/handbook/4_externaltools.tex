\section{External Tools}
\label{sc_externaltools}

\subsection{Subversion (svn)}
Subversion is a software versioning and revision control system. We use Subversion to
manage the source code of \Dumux, archive changes and central storage.
Basic svn commands are:
\begin{itemize}
  \item \texttt{svn add} to add files/folder to the repository.
        Use \texttt{svn add --depth=empty FOLDER} to add just the folder
        and \texttt{svn add YOURFILES} to add files. Generally, you should only add
        necessary text-based files. Please do not upload (large) binary files.
  \item \texttt{svn checkout} checkout a repository
  \item \texttt{svn update} updates files/folder
  \item \texttt{svn status} to track changes:
        \textbf{M}odified, \textbf{D}eleted, \textbf{A}dded, \textbf{?} not in repository
  \item \texttt{svn diff} to see the actual changes of files/folder
  \item \texttt{svn commit} upload changes to the repository
\end{itemize}
There are also tools providing a graphical user interface, like \emph{kdesvn} or \emph{eclipse}.

\subsection{Git}
Git is an other version control tool which is currently used to manage the \Dune modules.
The basic Git commands are:
\begin{itemize}
  \item \texttt{git checkout} receive a specified branch from the repository
  \item \texttt{git clone} clone a repository (similar to svn checkout)
  \item \texttt{git diff} to see the actual changes of a file/folder
  \item \texttt{git pull} pull changes from the repository (similar to svn update)
  \item \texttt{git status} to check which files/folders have been changed
\end{itemize}

\subsection{Eclipse}
There is an eclipse style file which can be used for Dumux.
\begin{enumerate}
  \item in eclipse open: \texttt{Window} $\rightarrow$ \texttt{Preferences} $\rightarrow$
        \texttt{C/C++}  $\rightarrow$ \texttt{Code Style} $\rightarrow$ \texttt{Formatter}
  \item press the \texttt{Import} button
  \item choose the file \texttt{eclipse\_profile.xml} from your dumux-devel directory
  \item make sure that that now \Dumux is chosen in \texttt{Select a profile}
\end{enumerate}

\subsection{Kate}
For kate there is syntax highlighting style for \Dumux input files. Simply
copy the file \texttt{dumux-devel/dumuxInputFiles.xml} to the \texttt{syntax} folder in
your kate configuration directory (e.g.
\texttt{HOME/.kde4/share/apps/katepart/syntax/dumuxInputFiles.xml}).

\subsection{ParaView}
\paragraph{Reload Button:}
There are scripts to reload \texttt{*.pvd} or series of {\texttt{*.vtu} files since ParaView 4.2.
The scripts can be found
\href{http://markmail.org/message/exxynsgishbvtngg#query:+page:1+mid:rxlwxs7uqrfgibyv+state:results}{\texttt{under this lin}}.
Just save the specific code portion in a file and load it via \texttt{Macros} $\rightarrow$ \texttt{Add new macro}.

\paragraph{Guide:}
Since ParaView 4.3.1 The ParaView Guide is partly
available for free download, see \url{http://www.paraview.org/documentation/}.
It corresponds to the ParaView book, only without three application chapters.
Attention, its size is 180 MiB.
