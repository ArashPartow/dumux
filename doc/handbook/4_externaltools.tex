\section{External Tools}
\label{sc_externaltools}

\subsection{Git}
Git is a version control tool which we use.
The basic Git commands are:
\begin{itemize}
  \item \texttt{git checkout} receive a specified branch from the repository
  \item \texttt{git clone} clone a repository; creates a local copy
  \item \texttt{git diff} to see the actual changes compared to your last commit
  \item \texttt{git pull} pull changes from the repository; synchronizes the
  repository with your local copy
  \item \texttt{git push} push comitted changes to the repository;  synchronizes
  your local copy with the repository
  \item \texttt{git status} to check which files/folders have been changed
  \item \texttt{git gui} graphical user interface, helps selecting changes for
  a commit
\end{itemize}

\subsection{Eclipse}
There is an Eclipse style file which can be used for \Dumux.
\begin{enumerate}
  \item open in eclipse: \texttt{Window} $\rightarrow$ \texttt{Preferences} $\rightarrow$
        \texttt{C/C++}  $\rightarrow$ \texttt{Code Style} $\rightarrow$ \texttt{Formatter}
  \item press the \texttt{Import} button
  \item choose the file \texttt{eclipse\_profile.xml} from your dumux-devel directory
  \item make sure that that now \Dumux is chosen in \texttt{Select a profile}
\end{enumerate}

\subsection{Kate}
For kate there is syntax highlighting style for \Dumux input files. Simply
copy the file \texttt{dumux-devel/dumux\-InputFiles.xml} to the \texttt{syntax} folder in
your kate configuration directory (e.g.
\texttt{HOME/.kde4/share/apps\-/katepart/syntax/dumuxInputFiles.xml}).

\subsection{ParaView}
\paragraph{Reload Button:}
There are scripts to reload \texttt{*.pvd} or series of {\texttt{*.vtu} files since ParaView 4.2.
The scripts can be found
\href{http://markmail.org/message/exxynsgishbvtngg#query:+page:1+mid:rxlwxs7uqrfgibyv+state:results}{\texttt{under this lin}}.
Just save the specific code portion in a file and load it via \texttt{Macros} $\rightarrow$ \texttt{Add new macro}.

\paragraph{Guide:}
Since ParaView 4.3.1 The ParaView Guide is partly
available for free download, see \url{http://www.paraview.org/documentation/}.
It corresponds to the ParaView book, only without three application chapters.
Attention, its size is 180 MiB.
