\section{Temporal Discretization and Solution Strategies}
In this section, the temporal discretization as well as solution strategies (monolithic/sequential) are presented.

\subsection{Temporal discretization}

Our systems of partial differential equations are discretized in space and in time.

Let us consider the general case of a balance equation of the following form
\begin{equation}\label{eq:generalbalance}
\frac{\partial m(u)}{\partial t} + \nabla\cdot\mathbf{f}(u, \nabla u) + q(u) = 0,
\end{equation}
seeking an unknown quantity $u$ in terms of storage $m$, flux $\mathbf{f}$ and source $q$.
All available Dumux models can be written mathematically in form of \eqref{eq:generalbalance}
with possibly vector-valued quantities $u$, $m$, $q$ and a tensor-valued flux $\mathbf{f}$.
For the sake of simplicity, we assume scalar quantities $u$, $m$, $q$ and a vector-valued
flux $\mathbf{f}$ in the notation below.

For discretizing \eqref{eq:generalbalance}, we need to choose an
approximation for the temporal derivative $\partial m(u)/\partial t$.
While many elaborate methods for this approximation exist,
we focus on the simplest one of a first order difference quotient
\begin{equation}\label{eq:euler}
\frac{\partial m(u_{k/k+1})}{\partial t}
\approx \frac{m(u_{k+1}) - m(u_k)}{\Delta t_{k+1}}
\end{equation}
for approximating the solution $u$ at time $t_k$ (forward) or $t_{k+1}$ (backward).
The question of whether to choose the forward or the backward quotient leads to the
explicit and implicit Euler method, respectively.
In case of the former, inserting \eqref{eq:euler} in \eqref{eq:generalbalance}
at time $t_k$ leads to
\begin{equation}\label{eq:expliciteuler}
\frac{m(u_{k+1}) - m(u_k)}{\Delta t_{k+1}} + \nabla\cdot\mathbf{f}(u_k, \nabla u_k) + q(u_k) = 0,
\end{equation}
whereas the implicit Euler method is described as
\begin{equation}\label{eq:impliciteuler}
\frac{m(u_{k+1}) - m(u_k)}{\Delta t_{k+1}}
+ \nabla\cdot\mathbf{f}(u_{k+1}, \nabla u_{k+1}) + q(u_{k+1}) = 0.
\end{equation}
Once the solution $u_k$ at time $t_k$ is known, it is straightforward
to determine $m(u_{k+1})$ from \eqref{eq:expliciteuler},
while attempting to do the same based on \eqref{eq:impliciteuler}
involves the solution of a system of equations.
On the other hand, the explicit method \eqref{eq:expliciteuler} is stable only
if the time step size $\Delta t_{k+1}$ is below a certain limit that depends
on the specific balance equation, whereas the implicit method \eqref{eq:impliciteuler}
is unconditionally stable.

\subsection{Solution strategies to solve equations}
The governing equations of each model can be solved monolithically or sequentially.
The basic idea of the sequential algorithm is to reformulate the
equations of multi-phase flow into one equation for
pressure and equations for phase/component/... transport. The pressure equation
is the sum of the mass balance equations and thus considers the total flow of the
fluid system. The new set of equations is considered as decoupled (or weakly coupled)
and can thus be solved sequentially. The most popular sequential model is the
fractional flow formulation for two-phase flow which is usually implemented applying
an IMplicit Pressure Explicit Saturation algorithm (IMPES).
In comparison to solving the equations monolithically, the sequential structure allows the use of
different discretization methods for the different equations. The standard method
used in the sequential algorithm is a cell-centered finite volume method. Further schemes,
so far only available for the two-phase pressure equation, are cell-centered finite
volumes with multi-point flux approximation (Mpfa-O method) and mimetic finite differences.
An $h$-adaptive implementation of both sequential algorithms is provided for two dimensions.
