\section{Prerequisites} \label{sec:prerequisites}
For this quick start guide the following software packages are required:
\begin{itemize}
\item GitLab client
\item A standard compliant C++ compiler supporting C++11 and the C++14 feature set of GCC 4.9. We support GCC 4.9 or newer and Clang 3.8 or newer.
\item CMake 2.8.12 or newer
\item pkg-config
\item ParaView (to visualize the results)
\end{itemize}

\section{Obtaining code and configuring all modules with a script}
We provide you with a shell-script \texttt{installDumux.sh} that facilitates setting up a {\Dune}/{\Dumux} directory tree
and configures all modules with CMake.
Copy the following lines into a text file named \texttt{installDumux.sh}:
\lstinputlisting[style=DumuxCode, numbersep=5pt, firstline=1, firstnumber=1]{installDumux.sh}

Place the \texttt{installDumux.sh} script in the directory where you want to install \Dumux and \Dune (a single
root folder \texttt{DUMUX} will be produced, so you do not need to provide one). Make \texttt{installDumux.sh} executable and run the script by typing into the terminal: \texttt{./installDumux.sh}

Configuring \Dune and \Dumux is done by the command-line script \texttt{dunecontrol}
using optimized configure options, see the line entitled \texttt{\# run build} in the \texttt{installDumux.sh} script.
More details about the build-system can be found in section \ref{buildIt}.

\subsection{A first test run of \Dumux}
When the \texttt{installDumux.sh} script from the subsection above has run successfully, you can execute a second script that
will compile and run a simple one-phase ground water flow example and will visualize the result using ParaView.
The test script can be obtained by copying the following lines into a text file named \texttt{test\_dumux.sh}
that has to be located in the same directory as the installation script.
\begin{lstlisting}[style=DumuxCode]
cd DUMUX/dumux/build-cmake/test/porousmediumflow/1p/implicit/isothermal
make -B test_1p_tpfa
./test_1p_tpfa params.input
paraview *pvd
\end{lstlisting}
After making \texttt{test\_dumux.sh} executable, it can be executed by typing into the terminal: \texttt{./test\_dumux.sh}.
If everything works fine, a ParaView window with the result should open automatically, showing the initial
conditions. Advance ParaView to the next frame (green arrow button) and rescale to data range (green double arrow on top right) to admire
the colorful pressure distribution.
