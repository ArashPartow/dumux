\section{Quick Installation of \Dumux}
\label{quick-install}

This only provides one quick way of installing \Dumux.
You should have a recent working Linux environment, no \Dune core modules should be installed.
If you need more information,
have \Dune already installed, please have a look at the detailed installation
instructions in Section \ref{install}.

\subsection{Obtaining the Code with the Script \texttt{checkout-dumux}}

The shell-script \texttt{checkout-dumux} facilitates setting up a {\Dune}/{\Dumux} directory tree.
It is available at \cite{DUMUX-DOWNLOAD}.
For example the second line below will check out the required \Dune modules and \texttt{dumux},
\texttt{dumux-devel} and the \texttt{external} folder, which contains some useful external software and libraries.
Again,  \texttt{joeuser} needs to be replaced by the actual user name.
\begin{lstlisting}[style=Bash]
$ checkout-dumux -h      # show help,
$ checkout-dumux -gme -u joeuser -p password -d DUMUX
\end{lstlisting}

Be aware that you cannot get \texttt{dumux-devel} or the external libraries from \texttt{dumux-external} unless
you have an SVN account to our servers.

If you want to install \Dune and \Dumux without the help of \texttt{checkout-dumux} script a complete installation
guide can be found in chapter \ref{install} or on the \Dune website \cite{DUNE-INST}.

\subsection{Build of \Dune and \Dumux}
\label{buildIt}
Building of \Dune and \Dumux is done by the command-line script \texttt{dunecontrol} as described in \Dune Installation Notes \cite{DUNE-INST}
and in much more comprehensive form in the \Dune Buildsystem Howto \cite{DUNE-BS}.
If something fails during the execution of \texttt{dunecontrol} feel free to report it to the \Dune or \Dumux developer mailing list,
but also try to include error details.

It is possible to compile \Dumux with nearly no explicit options to the build system.
However, for the successful compilation of \Dune and \Dumux, it is currently necessary to pass the
the option \texttt{-fno-strict-aliasing} to the \Cplusplus compiler
\cite{WIKIPED-ALIASING}, which is done here via a command-line argument to \texttt{dunecontrol}:
\begin{lstlisting}[style=Bash]
$ # make sure you are in the directory DUNE-Root
$ ./dune-common/bin/dunecontrol --configure-opts="CXXFLAGS=-fno-strict-aliasing" --use-cmake all
\end{lstlisting}

Too many options can make life hard. That's why usually option files are being used together with \texttt{dunecontrol} and its sub-tools.
Larger sets of options are kept in them. If you are going to compile with options suited for debugging the code, the following
can be a starting point:
\begin{lstlisting}[style=Bash]
$ # make sure you are in the directory DUNE-Root
$ cp dumux/debug.opts my-debug.opts      # create a personal version
$ gedit my-debug.opts                    # optional editing the options file
$ ./dune-common/bin/dunecontrol --opts=my-debug.opts --use-cmake all
\end{lstlisting}

More optimized code, which is typically not usable for standard debugging tasks, can be produced by
\begin{lstlisting}[style=Bash]
$ cp dumux/optim.opts my-optim.opts
$ ./dune-common/bin/dunecontrol --opts=my-optim.opts --use-cmake all
\end{lstlisting}

Sometimes it is necessary to have additional options which
are specific to a package set of an operating system or
sometimes you have your own preferences.
Feel free to work with your own set of options, which may evolve over time.
The option files above are to be understood more as a starting point
for setting up an own customization than as something which is fixed.
The use of external libraries can make it necessary to add quite many options in an option-file.
It can be helpful to give your customized option file its own name, as done above.
One avoids confusing it with the option files which came out of the distribution
and which can be possibly updated by subversion later on.

