%%%%%%%%%%%%%%%%%%%%%%%%%%%%%%%%%%%%%%%%%%%%%%%%%%%%%%%%%%%%%%%%%
% This file has been autogenerated from the LaTeX part of the   %
% doxygen documentation; DO NOT EDIT IT! Change the model's .hh %
% file instead!!                                                %
%%%%%%%%%%%%%%%%%%%%%%%%%%%%%%%%%%%%%%%%%%%%%%%%%%%%%%%%%%%%%%%%%

See \hyperlink{a00633}{Two\-P\-Two\-C\-Model} for reference to the equations used. The \hyperlink{a00074}{C\-O2} model is derived from the 2p2c model. In the \hyperlink{a00074}{C\-O2} model the phase switch criterion is different from the 2p2c model. The phase switch occurs when the equilibrium concentration of a component in a phase is exceeded, instead of the sum of the components in the virtual phase (the phase which is not present) being greater that unity as done in the 2p2c model. The \hyperlink{a00078}{C\-O2\-Volume\-Variables} do not use a constraint solver for calculating the mole fractions as is the case in the 2p2c model. Instead mole fractions are calculated in the Fluid\-System with a given temperature, pressurem and salinity. The model is able to use either mole or mass fractions. The property use\-Moles can be set to either true or false in the problem file. Make sure that the according units are used in the problem setup. use\-Moles is set to false by default.

