%%%%%%%%%%%%%%%%%%%%%%%%%%%%%%%%%%%%%%%%%%%%%%%%%%%%%%%%%%%%%%%%%
% This file has been autogenerated from the LaTeX part of the   %
% doxygen documentation; DO NOT EDIT IT! Change the model's .hh %
% file instead!!                                                %
%%%%%%%%%%%%%%%%%%%%%%%%%%%%%%%%%%%%%%%%%%%%%%%%%%%%%%%%%%%%%%%%%

This model implements a two-\/phase flow of compressible immiscible fluids $\alpha \in \{ w, n \}$. The deformation of the solid matrix is described with a quasi-\/stationary momentum balance equation. The influence of the pore fluid is accounted for through the effective stress concept (Biot 1941). The total stress acting on a rock is partially supported by the rock matrix and partially supported by the pore fluid. The effective stress represents the share of the total stress which is supported by the solid rock matrix and can be determined as a function of the strain according to Hooke\textquotesingle{}s law.

As an equation for the conservation of momentum within the fluid phases the standard multiphase Darcy\textquotesingle{}s approach is used\-: \[ v_\alpha = - \frac{k_{r\alpha}}{\mu_\alpha} \textbf{K} \left(\textbf{grad}\, p_\alpha - \varrho_{\alpha} {\textbf g} \right) \]

Gravity can be enabled or disabled via the property system. By inserting this into the continuity equation, one gets \[ \frac{\partial \phi_{eff} \varrho_\alpha S_\alpha}{\partial t} - \text{div} \left\{ \varrho_\alpha \frac{k_{r\alpha}}{\mu_\alpha} \mathbf{K}_\text{eff} \left(\textbf{grad}\, p_\alpha - \varrho_{\alpha} \mathbf{g} \right) - \phi_{eff} \varrho_\alpha S_\alpha \frac{\partial \mathbf{u}}{\partial t} \right\} - q_\alpha = 0 \;, \]

A quasi-\/stationary momentum balance equation is solved for the changes with respect to the initial conditions (Darcis 2012), note that this implementation assumes the soil mechanics sign convention (i.\-e. compressive stresses are negative)\-: \[ \text{div}\left( \boldsymbol{\Delta \sigma'}- \Delta p_{eff} \boldsymbol{I} \right) + \Delta \varrho_b {\textbf g} = 0 \;, \] with the effective stress\-: \[ \boldsymbol{\sigma'} = 2\,G\,\boldsymbol{\epsilon} + \lambda \,\text{tr} (\boldsymbol{\epsilon}) \, \mathbf{I}. \]

and the strain tensor $\boldsymbol{\epsilon}$ as a function of the solid displacement gradient $\textbf{grad} \mathbf{u}$\-: \[ \boldsymbol{\epsilon} = \frac{1}{2} \, (\textbf{grad} \mathbf{u} + \textbf{grad}^T \mathbf{u}). \]

Here, the rock mechanics sign convention is switch off which means compressive stresses are $<$ 0 and tensile stresses are $>$ 0. The rock mechanics sign convention can be switched on for the vtk output via the property system.

The effective porosity and the effective permeability are calculated as a function of the solid displacement\-: \[ \phi_{eff} = \frac{\phi_{init} + \text{div} \mathbf{u}}{1 + \text{div} \mathbf{u}} \] \[ K_{eff} = K_{init} \text{exp}\left( 22.2(\phi_{eff}/\phi_{init} -1 )\right) \] The mass balance equations are discretized using a vertex-\/centered finite volume (box) or cell-\/centered finite volume scheme as spatial and the implicit Euler method as time discretization. The momentum balance equations are discretized using a standard Galerkin Finite Element method as spatial discretization scheme.

The primary variables are the wetting phase pressure $p_w$, the nonwetting phase saturation $S_n$ and the solid displacement vector $\mathbf{u}$ (changes in solid displacement with respect to initial conditions).

