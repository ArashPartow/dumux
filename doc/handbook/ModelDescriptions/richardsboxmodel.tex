%%%%%%%%%%%%%%%%%%%%%%%%%%%%%%%%%%%%%%%%%%%%%%%%%%%%%%%%%%%%%%%%%
% This file has been autogenerated from the LaTeX part of the   %
% doxygen documentation; DO NOT EDIT IT! Change the model's .hh %
% file instead!!                                                %
%%%%%%%%%%%%%%%%%%%%%%%%%%%%%%%%%%%%%%%%%%%%%%%%%%%%%%%%%%%%%%%%%

In the unsaturated zone, Richards' equation \[ \frac{\partial\;\phi S_w \rho_w}{\partial t} - \text{div} \left( \rho_w \frac{k_{rw}}{\mu_w} \; \mathbf{K} \; \text{\textbf{grad}}\left( p_w - g\rho_w \right) \right) = q_w, \] is frequently used to approximate the water distribution above the groundwater level.

It can be derived from the two-\/phase equations, i.\-e. \[ \frac{\partial\;\phi S_\alpha \rho_\alpha}{\partial t} - \text{div} \left( \rho_\alpha \frac{k_{r\alpha}}{\mu_\alpha}\; \mathbf{K} \; \text{\textbf{grad}}\left( p_\alpha - g\rho_\alpha \right) \right) = q_\alpha, \] where $\alpha \in \{w, n\}$ is the fluid phase, $\rho_\alpha$ is the fluid density, $S_\alpha$ is the fluid saturation, $\phi$ is the porosity of the soil, $k_{r\alpha}$ is the relative permeability for the fluid, $\mu_\alpha$ is the fluid's dynamic viscosity, $\mathbf{K}$ is the intrinsic permeability, $p_\alpha$ is the fluid pressure and $g$ is the potential of the gravity field.

In contrast to the full two-\/phase model, the Richards model assumes gas as the non-\/wetting fluid and that it exhibits a much lower viscosity than the (liquid) wetting phase. (For example at atmospheric pressure and at room temperature, the viscosity of air is only about $1\%$ of the viscosity of liquid water.) As a consequence, the $\frac{k_{r\alpha}}{\mu_\alpha}$ term typically is much larger for the gas phase than for the wetting phase. For this reason, the Richards model assumes that $\frac{k_{rn}}{\mu_n}$ is infinitly large. This implies that the pressure of the gas phase is equivalent to the static pressure distribution and that therefore, mass conservation only needs to be considered for the wetting phase.

The model thus choses the absolute pressure of the wetting phase $p_w$ as its only primary variable. The wetting phase saturation is calculated using the inverse of the capillary pressure, i.\-e. \[ S_w = p_c^{-1}(p_n - p_w) \] holds, where $p_n$ is a given reference pressure. Nota bene, that the last step is assumes that the capillary pressure-\/saturation curve can be uniquely inverted, so it is not possible to set the capillary pressure to zero when using the Richards model!

