%%%%%%%%%%%%%%%%%%%%%%%%%%%%%%%%%%%%%%%%%%%%%%%%%%%%%%%%%%%%%%%%%
% This file has been autogenerated from the LaTeX part of the   %
% doxygen documentation; DO NOT EDIT IT! Change the model's .hh %
% file instead!!                                                %
%%%%%%%%%%%%%%%%%%%%%%%%%%%%%%%%%%%%%%%%%%%%%%%%%%%%%%%%%%%%%%%%%

This model implements three-phase three-component flow of three fluid phases $\alpha \in \{ \text{water}, \text{gas}, \text{NAPL} \}$ each composed of up to three components $\kappa \in \{ \text{water}, \text{air}, \text{contaminant} \}$. The standard multiphase Darcy approach is used as the equation for the conservation of momentum: \[ v_\alpha = - \frac{k_{r\alpha}}{\mu_\alpha} \mbox{\bf K} \left(\text{grad}\, p_\alpha - \varrho_{\alpha} \mbox{\bf g} \right) \]

\-By inserting this into the equations for the conservation of the components, one transport equation for each component is obtained as \begin{eqnarray*} && \phi \frac{\partial (\sum_\alpha \varrho_{\text{mol}, \alpha} x_\alpha^\kappa S_\alpha )}{\partial t} - \sum\limits_\alpha \text{div} \left\{ \frac{k_{r\alpha}}{\mu_\alpha} \varrho_{\text{mol}, \alpha} x_\alpha^\kappa \mbox{\bf K} (\text{grad}\, p_\alpha - \varrho_{\text{mass}, \alpha} \mbox{\bf g}) \right\} \nonumber \\ \nonumber \\ && - \sum\limits_\alpha \text{div} \left\{ D_{pm}^\kappa \varrho_{\text{mol}, \alpha } \text{grad}\, x_\alpha^\kappa \right\} - q^\kappa = 0 \qquad \forall \kappa , \; \forall \alpha \end{eqnarray*}

\-Note that these balance equations are molar.

\-The equations are discretized using a fully-\/coupled vertex centered finite volume (\-B\-O\-X) scheme as spatial scheme and the implicit \-Euler method as temporal discretization.

\-The model uses commonly applied auxiliary conditions like $S_w + S_n + S_g = 1$ for the saturations and $x^w_\alpha + x^a_\alpha + x^c_\alpha = 1$ for the mole fractions. \-Furthermore, the phase pressures are related to each other via capillary pressures between the fluid phases, which are functions of the saturation, e.\-g. according to the approach of \-Parker et al.

\-The used primary variables are dependent on the locally present fluid phases \-An adaptive primary variable switch is included. \-The phase state is stored for all nodes of the system. \-The following cases can be distinguished\-:
\begin{itemize}
\item \-All three phases are present\-: \-Primary variables are two saturations $(S_w$ and $S_n)$, and a pressure, in this case $p_g$.
\item \-Only the water phase is present\-: \-Primary variables are now the mole fractions of air and contaminant in the water phase $(x_w^a$ and $x_w^c)$, as well as the gas pressure, which is, of course, in a case where only the water phase is present, just the same as the water pressure.
\item \-Gas and \-N\-A\-P\-L phases are present\-: \-Primary variables $(S_n$, $x_g^w$, $p_g)$.
\item \-Water and \-N\-A\-P\-L phases are present\-: \-Primary variables $(S_n$, $x_w^a$, $p_g)$.
\item \-Only gas phase is present\-: \-Primary variables $(x_g^w$, $x_g^c$, $p_g)$.
\item \-Water and gas phases are present\-: \-Primary variables $(S_w$, $x_w^g$, $p_g)$.
\end{itemize}

