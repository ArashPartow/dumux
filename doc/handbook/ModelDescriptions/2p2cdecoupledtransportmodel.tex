%%%%%%%%%%%%%%%%%%%%%%%%%%%%%%%%%%%%%%%%%%%%%%%%%%%%%%%%%%%%%%%%%
% This file has been autogenerated from the LaTeX part of the   %
% doxygen documentation; DO NOT EDIT IT! Change the model's .hh %
% file instead!!                                                %
%%%%%%%%%%%%%%%%%%%%%%%%%%%%%%%%%%%%%%%%%%%%%%%%%%%%%%%%%%%%%%%%%

The transport step is described by the finite volume model for the solution of the transport equation for compositional two-\/phase flow. \[ \frac{\partial C^\kappa}{\partial t} = - \nabla \cdot \left( \sum_{\alpha} X^{\kappa}_{\alpha} \varrho_{alpha} \bf{v}_{\alpha}\right) + q^{\kappa}, \] where $ \bf{v}_{\alpha} = - \lambda_{\alpha} \bf{K} \left(\nabla p_{\alpha} + \rho_{\alpha} \bf{g} \right) $. $ p_{\alpha} $ denotes the phase pressure, $ \bf{K} $ the absolute permeability, $ \lambda_{\alpha} $ the phase mobility, $ \rho_{\alpha} $ the phase density and $ \bf{g} $ the gravity constant and $ C^{\kappa} $ the total \hyperlink{a00070}{Component} concentration. The whole flux contribution for each cell is subdivided into a storage term, a flux term and a source term. Corresponding functions ({\ttfamily \hyperlink{a00145_a13998fc22be58456c4bf8e3f4b12d89c}{get\-Flux()}} and {\ttfamily \hyperlink{a00145_a40fc97d83d3d15cdd29574d3a38fdafb}{get\-Flux\-On\-Boundary()}}) are provided, internal sources are directly treated.


