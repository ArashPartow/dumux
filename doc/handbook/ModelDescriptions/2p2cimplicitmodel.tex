%%%%%%%%%%%%%%%%%%%%%%%%%%%%%%%%%%%%%%%%%%%%%%%%%%%%%%%%%%%%%%%%%
% This file has been autogenerated from the LaTeX part of the   %
% doxygen documentation; DO NOT EDIT IT! Change the model's .hh %
% file instead!!                                                %
%%%%%%%%%%%%%%%%%%%%%%%%%%%%%%%%%%%%%%%%%%%%%%%%%%%%%%%%%%%%%%%%%

This model implements two-\/phase two-\/component flow of two compressible and partially miscible fluids $\alpha \in \{ w, n \}$ composed of the two components $\kappa \in \{ w, a \}$. The standard multiphase Darcy approach is used as the equation for the conservation of momentum\-: \[ v_\alpha = - \frac{k_{r\alpha}}{\mu_\alpha} \mathbf{K} \left(\textbf{grad}\, p_\alpha - \varrho_{\alpha} \mbox{\bf g} \right) \]

By inserting this into the equations for the conservation of the components, one gets one transport equation for each component \begin{eqnarray*} && \phi \frac{\partial (\sum_\alpha \varrho_\alpha \frac{M^\kappa}{M_\alpha} x_\alpha^\kappa S_\alpha )} {\partial t} - \sum_\alpha \text{div} \left\{ \varrho_\alpha \frac{M^\kappa}{M_\alpha} x_\alpha^\kappa \frac{k_{r\alpha}}{\mu_\alpha} \mathbf{K} (\textbf{grad}\, p_\alpha - \varrho_{\alpha} \mbox{\bf g}) \right\} \nonumber \\ \nonumber \\ &-& \sum_\alpha \text{div} \left\{ D_{\alpha,\text{pm}}^\kappa \varrho_{\alpha} \frac{M^\kappa}{M_\alpha} \textbf{grad} x^\kappa_{\alpha} \right\} - \sum_\alpha q_\alpha^\kappa = 0 \qquad \kappa \in \{w, a\} \, , \alpha \in \{w, g\} \end{eqnarray*}

All equations are discretized using a vertex-\/centered finite volume (box) or cell-\/centered finite volume scheme as spatial and the implicit Euler method as time discretization.

By using constitutive relations for the capillary pressure $p_c = p_n - p_w$ and relative permeability $k_{r\alpha}$ and taking advantage of the fact that $S_w + S_n = 1$ and $x^\kappa_w + x^\kappa_n = 1$, the number of unknowns can be reduced to two. The used primary variables are, like in the two-\/phase model, either $p_w$ and $S_n$ or $p_n$ and $S_w$. The formulation which ought to be used can be specified by setting the {\ttfamily Formulation} property to either Two\-P\-Two\-C\-Indices\-::p\-Ws\-N or Two\-P\-Two\-C\-Indices\-::p\-Ns\-W. By default, the model uses $p_w$ and $S_n$. Moreover, the second primary variable depends on the phase state, since a primary variable switch is included. The phase state is stored for all nodes of the system. The model is able to use either mole or mass fractions. The property use\-Moles can be set to either true or false in the problem file. Make sure that the according units are used in the problem setup. use\-Moles is set to true by default. Following cases can be distinguished\-:
\begin{itemize}
\item Both phases are present\-: The saturation is used (either $S_n$ or $S_w$, dependent on the chosen {\ttfamily Formulation}), as long as $ 0 < S_\alpha < 1$.
\item Only wetting phase is present\-: The mole fraction of, e.\-g., air in the wetting phase $x^a_w$ is used, as long as the maximum mole fraction is not exceeded $(x^a_w<x^a_{w,max})$
\item Only non-\/wetting phase is present\-: The mole fraction of, e.\-g., water in the non-\/wetting phase, $x^w_n$, is used, as long as the maximum mole fraction is not exceeded $(x^w_n<x^w_{n,max})$
\end{itemize}

