%%%%%%%%%%%%%%%%%%%%%%%%%%%%%%%%%%%%%%%%%%%%%%%%%%%%%%%%%%%%%%%%%
% This file has been autogenerated from the LaTeX part of the   %
% doxygen documentation; DO NOT EDIT IT! Change the model's .hh %
% file instead!!                                                %
%%%%%%%%%%%%%%%%%%%%%%%%%%%%%%%%%%%%%%%%%%%%%%%%%%%%%%%%%%%%%%%%%

This model implements a generic energy balance for single and multi-\/phase transport problems. Currently the non-\/isothermal model can be used on top of the 1p2c, 2p, 2p2c and 3p3c models. Comparison to simple analytical solutions for pure convective and conductive problems are found in the 1p2c test. Also refer to this test for details on how to activate the non-\/isothermal model.

For the energy balance, local thermal equilibrium is assumed. This results in one energy conservation equation for the porous solid matrix and the fluids\-: \begin{align*} \phi \frac{\partial \sum_\alpha \varrho_\alpha u_\alpha S_\alpha}{\partial t} & + \left( 1 - \phi \right) \frac{\partial (\varrho_s c_s T)}{\partial t} - \sum_\alpha \text{div} \left\{ \varrho_\alpha h_\alpha \frac{k_{r\alpha}}{\mu_\alpha} \mathbf{K} \left( \textbf{grad}\,p_\alpha - \varrho_\alpha \mbox{\bf g} \right) \right\} \\ & - \text{div} \left(\lambda_{pm} \textbf{grad} \, T \right) - q^h = 0. \end{align*} where $h_\alpha$ is the specific enthalpy of a fluid phase $\alpha$ and $u_\alpha = h_\alpha - p_\alpha/\varrho_\alpha$ is the specific internal energy of the phase.

