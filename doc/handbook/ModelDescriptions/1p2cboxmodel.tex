%%%%%%%%%%%%%%%%%%%%%%%%%%%%%%%%%%%%%%%%%%%%%%%%%%%%%%%%%%%%%%%%%
% This file has been autogenerated from the LaTeX part of the   %
% doxygen documentation; DO NOT EDIT IT! Change the model's .hh %
% file instead!!                                                %
%%%%%%%%%%%%%%%%%%%%%%%%%%%%%%%%%%%%%%%%%%%%%%%%%%%%%%%%%%%%%%%%%

This model implements a one-\/phase flow of a compressible fluid, that consists of two components, using a standard Darcy approach as the equation for the conservation of momentum\-: \[ v_{D} = - \frac{\textbf K}{\mu} \left(\text{grad} p - \varrho {\textbf g} \right) \]

Gravity can be enabled or disabled via the property system. By inserting this into the continuity equation, one gets \[ \Phi \frac{\partial \varrho}{\partial t} - \text{div} \left\{ \varrho \frac{\textbf K}{\mu} \left(\text{grad}\, p - \varrho {\textbf g} \right) \right\} = q \;, \]

The transport of the components is described by the following equation\-: \[ \Phi \frac{ \partial \varrho x}{\partial t} - \text{div} \left( \varrho \frac{{\textbf K} x}{\mu} \left( \text{grad}\, p - \varrho {\textbf g} \right) + \varrho \tau \Phi D \text{grad} x \right) = q. \]

All equations are discretized using a fully-\/coupled vertex-\/centered finite volume (box) scheme as spatial and the implicit Euler method as time discretization.

The primary variables are the pressure $p$ and the mole or mass fraction of dissolved component $x$.

