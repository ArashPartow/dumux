%%%%%%%%%%%%%%%%%%%%%%%%%%%%%%%%%%%%%%%%%%%%%%%%%%%%%%%%%%%%%%%%%
% This file has been autogenerated from the LaTeX part of the   %
% doxygen documentation; DO NOT EDIT IT! Change the model's .hh %
% file instead!!                                                %
%%%%%%%%%%%%%%%%%%%%%%%%%%%%%%%%%%%%%%%%%%%%%%%%%%%%%%%%%%%%%%%%%

This model implements three-\/phase three-\/component flow of three fluid phases $\alpha \in \{ \text{water, gas, NAPL} \}$ each composed of up to three components $\kappa \in \{ \text{water, air, contaminant} \}$. The standard multiphase Darcy approach is used as the equation for the conservation of momentum\-: \[ v_\alpha = - \frac{k_{r\alpha}}{\mu_\alpha} \mathbf{K} \left(\textbf{grad}\, p_\alpha - \varrho_{\alpha} \mbox{\bf g} \right) \]

By inserting this into the equations for the conservation of the components, one transport equation for each component is obtained as \begin{eqnarray*} && \phi \frac{\partial (\sum_\alpha \varrho_\alpha X_\alpha^\kappa S_\alpha )}{\partial t} - \sum\limits_\alpha \text{div} \left\{ \frac{k_{r\alpha}}{\mu_\alpha} \varrho_\alpha X_\alpha^\kappa \mathbf{K} (\textbf{grad}\; p_\alpha - \varrho_\alpha \mbox{\bf g}) \right\} \nonumber \\ \nonumber \\ && - \sum\limits_\alpha \text{div} \left\{ D_{\alpha,\text{pm}}^\kappa \varrho_\alpha \frac{M^\kappa}{M_\alpha} \textbf{grad} x^\kappa_{\alpha} \right\} - q^\kappa = 0 \qquad \forall \kappa , \; \forall \alpha \end{eqnarray*}

Note that these balance equations above are molar. In addition to that, a single balance of thermal energy is formulated for the fluid-\/filled porous medium under the assumption of local thermal equilibrium \begin{eqnarray*} && \phi \frac{\partial \left( \sum_\alpha \varrho_\alpha u_\alpha S_\alpha \right)}{\partial t} + \left( 1 - \phi \right) \frac{\partial (\varrho_s c_s T)}{\partial t} - \sum_\alpha \text{div} \left\{ \varrho_\alpha h_\alpha \frac{k_{r\alpha}}{\mu_\alpha} \mathbf{K} \left( \textbf{grad}\, p_\alpha - \varrho_\alpha \mathbf{g} \right) \right\} \\ &-& \text{div} \left( \lambda_{pm} \textbf{grad} \, T \right) - q^h = 0 \qquad \alpha \in \{w, n, g\} \end{eqnarray*}

All equations are discretized using a vertex-\/centered finite volume (box) or cell-\/centered finite volume scheme as spatial and the implicit Euler method as time discretization.

The model uses commonly applied auxiliary conditions like $S_w + S_n + S_g = 1$ for the saturations and $x^w_\alpha + x^a_\alpha + x^c_\alpha = 1$ for the mole fractions. Furthermore, the phase pressures are related to each other via capillary pressures between the fluid phases, which are functions of the saturation, e.\-g. according to the approach of Parker et al.

The used primary variables are dependent on the locally present fluid phases. An adaptive primary variable switch is included. The phase state is stored for all nodes of the system. The following cases can be distinguished\-:
\begin{itemize}
\item All three phases are present\-: Primary variables are two saturations $(S_w$ and $S_n)$, a pressure, in this case $p_g$, and the temperature $T$.
\item Only the water phase is present\-: Primary variables are now the mole fractions of air and contaminant in the water phase $(x_w^a$ and $x_w^c)$, as well as temperature and the gas pressure, which is, of course, in a case where only the water phase is present, just the same as the water pressure.
\item Gas and N\-A\-P\-L phases are present\-: Primary variables $(S_n$, $x_g^w$, $p_g$, $T)$.
\item Water and N\-A\-P\-L phases are present\-: Primary variables $(S_n$, $x_w^a$, $p_g$, $T)$.
\item Only gas phase is present\-: Primary variables $(x_g^w$, $x_g^c$, $p_g$, $T)$.
\item Water and gas phases are present\-: Primary variables $(S_w$, $x_w^g$, $p_g$, $T)$.
\end{itemize}

