%%%%%%%%%%%%%%%%%%%%%%%%%%%%%%%%%%%%%%%%%%%%%%%%%%%%%%%%%%%%%%%%%
% This file has been autogenerated from the LaTeX part of the   %
% doxygen documentation; DO NOT EDIT IT! Change the model's .hh %
% file instead!!                                                %
%%%%%%%%%%%%%%%%%%%%%%%%%%%%%%%%%%%%%%%%%%%%%%%%%%%%%%%%%%%%%%%%%

This model implements a $M$-\/phase flow of a fluid mixture composed of $N$ chemical species. The phases are denoted by lower index $\alpha \in \{ 1, \dots, M \}$. All fluid phases are mixtures of $N \geq M - 1$ chemical species which are denoted by the upper index $\kappa \in \{ 1, \dots, N \} $.

The momentum approximation can be selected via \char`\"{}\-Base\-Flux\-Variables\char`\"{}\-: Darcy (\hyperlink{a00278}{Implicit\-Darcy\-Flux\-Variables}) and Forchheimer (\hyperlink{a00279}{Implicit\-Forchheimer\-Flux\-Variables}) relations are available for all Box models.

By inserting this into the equations for the conservation of the mass of each component, one gets one mass-\/continuity equation for each component $\kappa$ \[ \sum_{\kappa} \left( \phi \frac{\partial \left(\varrho_\alpha x_\alpha^\kappa S_\alpha\right)}{\partial t} + \mathrm{div}\; \left\{ v_\alpha \frac{\varrho_\alpha}{\overline M_\alpha} x_\alpha^\kappa \right\} \right) = q^\kappa \] with $\overline M_\alpha$ being the average molar mass of the phase $\alpha$\-: \[ \overline M_\alpha = \sum_\kappa M^\kappa \; x_\alpha^\kappa \]

For the missing $M$ model assumptions, the model assumes that if a fluid phase is not present, the sum of the mole fractions of this fluid phase is smaller than $1$, i.\-e. \[ \forall \alpha: S_\alpha = 0 \implies \sum_\kappa x_\alpha^\kappa \leq 1 \]

Also, if a fluid phase may be present at a given spatial location its saturation must be positive\-: \[ \forall \alpha: \sum_\kappa x_\alpha^\kappa = 1 \implies S_\alpha \geq 0 \]

Since at any given spatial location, a phase is always either present or not present, one of the strict equalities on the right hand side is always true, i.\-e. \[ \forall \alpha: S_\alpha \left( \sum_\kappa x_\alpha^\kappa - 1 \right) = 0 \] always holds.

These three equations constitute a non-\/linear complementarity problem, which can be solved using so-\/called non-\/linear complementarity functions $\Phi(a, b)$ which have the property \[\Phi(a,b) = 0 \iff a \geq0 \land b \geq0 \land a \cdot b = 0 \]

Several non-\/linear complementarity functions have been suggested, e.\-g. the Fischer-\/\-Burmeister function \[ \Phi(a,b) = a + b - \sqrt{a^2 + b^2} \;. \] This model uses \[ \Phi(a,b) = \min \{a, b \}\;, \] because of its piecewise linearity.

These equations are then discretized using a fully-\/implicit vertex centered finite volume scheme (often known as 'box'-\/scheme) for spatial discretization and the implicit Euler method as temporal discretization.

The model assumes local thermodynamic equilibrium and uses the following primary variables\-:
\begin{itemize}
\item The component fugacities $f^1, \dots, f^{N}$
\item The pressure of the first phase $p_1$
\item The saturations of the first $M-1$ phases $S_1, \dots, S_{M-1}$
\item Temperature $T$ if the energy equation is enabled
\end{itemize}

