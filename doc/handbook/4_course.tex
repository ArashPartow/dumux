\section{Hands-on \Dumux experience -- the \Dumux course}
This course is offered once a year over a period of 3 days at the University of Stuttgart.
If you're looking for information on attending, subscribe to the \Dumux mailing list
and stay tuned for updates:
\url{https://listserv.uni-stuttgart.de/mailman/listinfo/dumux}. \par
%
\textit{``But the course won't take place for another 6 months!"} and,
\textit{``I want to start developing a numerical model of my challenging and
  interesting process now!"}, you think.
Not a problem. The course materials are all shared online in their own
git repository. A series of beginner-level exercises are explained
such that you can see how a model is developed in \Dumux. As a teaser, we've
 also included a suite of examples from hot topics we're working on. Models
  exploring ``Coupling free flow and porous-media flow", ``Flow in fractured
   porous media" and ``Fluid-solid phase change" are all introduced.  \par
 %
\textit{``Sounds great, but where is this material? I can't find it within
what I've downloaded."}, you question.
The \Dumux course material is available online:
\url{https://git.iws.uni-stuttgart.de/dumux-repositories/dumux-course}. \par
In order to download this repository, which acts as an additional module to
the \Dumux base, you can download an installation script with the following command:
\begin{lstlisting}[style=Bash]
$ wget https://git.iws.uni-stuttgart.de/dumux-repositories/dumux-course/raw/releases/3.2/scripts/install.sh
\end{lstlisting}
This script will install \texttt{dumux}, it's Dune dependencies, and the \texttt{dumux-course}
repository. Within the directory \texttt{dumux-course} there are a series of exercises
and slides describing the previously described examples. \par
%
The \Dumux course will be updated with each \Dumux release.
The above script will download the correct version (\textbf{releases/3.2}) of both
the \texttt{dumux} and \texttt{dumux-course} module.
