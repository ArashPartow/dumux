In \Dumux two sorts of models are implemented: Fully-coupled models and
decoupled models. In the fully-coupled models a flow system is described by a
system of strongly coupled equations, which can be for example mass balance
equations for phases, mass balance equations for components or energy balance
equations. In contrast, a decoupled model consists of a pressure equation, which
is iteratively coupled to a saturation equation, concentration equations, energy
balance equations, etc.

Examples for different kinds of both, coupled and decoupled models, are
isothermal two-phase models, isothermal two-phase two-component models,
non-isothermal two-phase models, and non-isothermal two-phase two-component
models.

The following two sections demonstrate solving problems using a fully-coupled
model \ref{tutorial-coupled} and a decoupled model \ref{tutorial-decoupled}.
An isothermal two-phase system (two fluid phases, one solid phase) will be
considered.
