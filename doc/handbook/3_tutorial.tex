In \Dumux two sorts of models are implemented: Fully-coupled models and
decoupled models. In the fully-coupled models a flow system is described by a
system of strongly coupled equations, which can be for example mass balance
equations for phases, mass balance equations for components or energy balance
equations. In contrast, a decoupled model consists of a pressure equation, which
is iteratively coupled to a saturation equation, concentration equations, energy
balance equations, etc.

Examples for different kinds of both, coupled and decoupled models, are
isothermal two-phase models, isothermal two-phase two-component models,
non-isothermal two-phase models and non-isothermal two-phase two-component
models.

In section \ref{box} a short introduction to the box method is given, in section
\ref{cc} the cell centered finite volume method is introduced. The box method is
used in the fully-coupled models for the spatial discretization of the system of
equations. The decoupled models employ usually a cell centered finite volume
scheme. The following two sections of the tutorial demonstrate how to solve
problems using, first, a fully-coupled model (section \ref{tutorial-coupled})
and, second, using a decoupled model (section \ref{tutorial-decoupled}). Being
the easiest case, an isothermal two-phase system (two fluid phases, one solid
phase) will be considered.

