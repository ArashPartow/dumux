\section{Grid Handling}
\label{sec:gridhandling}

This section summarizes some ideas about grid generation and grids that can be used by \Dumux. There are
several external grid generates available which can be used. The output files of these generators need
usually conversion into the Dune Grid Format (DGF), or some other format which can be read in by \Dune.
We intend to give brief ideas how to work with external grids. However, this list is not complete.

\subsection{DGF}
Most of our \Dumux tests and tutorials use the Dune Grid Format (DGF) to read in grids. A detailed description
of the DGF format and some examples can be found in the \Dune doxygen documentation
\textbf{(Modules $\rightarrow$ I/O $\rightarrow$ Dune Grid Format (DGF)}). To generate larger or more
complex DGF files, we recommend to write your own scripts, e.g in C++, Matlab or Python.

The DGF format can also used to read in spatial parameters defined on the grid. These parameters can
be defined on nodes as well as on the elements. An example for predefined parameters on a grid is
the \texttt{test\_boxco2} or \texttt{test\_cco2} in the  \texttt{dumux/test/co2} folder.

% Inside \Dumux, the \texttt{DGFGridCreater} is set by default and doesn't need to be set your problem file.


\subsection{GMSH}


\subsection{Petrel}


\subsection{ArtMesh}
\href{http://www.topologica.org/toplog/wp/}{ArtMesh} is a 3D mesh generation software. It has its own mesh file format
which can be read by \Dumux via the ArtGridCreator. Traditionally it was used within \Dumux for fracture simulations with
the discrete fracture matrix model (\texttt{2pdfm}). A detailed description of the fracture network creation and gridding
can be found for example in the dissertation of \href{http://elib.uni
-stuttgart.de/opus/frontdoor.php?source_opus=8047&la=de}{Tatomir}, pp. 68.

\subsection{ICEM}
\todo[inline]{Detailierte Beschreibung im Wiki? Links entfernen, die nicht funktionieren. Text überarbeiten. (Natalie)}
For complex geometries a graphical tool to create grids might be appropriate. One possibility to mesh for example CAD
geometry data is the commercial software \href{http://www.ansys.com/Products/Other+Products/ANSYS+ICEM+CFD/}{ANSYS ICEM
CFD}. A very detailed, but outdated description can be found at the LH2 internal wiki. A more recent best practice guide is available at
\url{XXX}. At LH2 exists a script which converts the ICEM mesh into the DGF.
