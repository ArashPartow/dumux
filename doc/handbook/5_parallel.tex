\section{Parallel Computation}
\label{sec:parallelcomputation}

\Dumux also support parallel computation. The parallel version needs an external MPI libary.

Posibilities are OpenMPI MPICH and IntelMPI.

Depending on the grid manager METIS or ParMETIS can also be used for paritioning.


In the following show  how to prepare a model an run it in parallel whith
the imcompressible 2p model.

dumux/test/porousmediumflow/2p/implicit/incompressible
 

\subsection{prepareing the model}

If the parallel AMGBackend is not allready set in your application
you should from the sequential solver backend to the parallel amg  backend 
in your application.

First include the header files for the parallel AMGBackend
#include <dumux/linear/amgbackend.hh>

and remove the header files of the sequential backend

#include <dumux/linear/seqsolverbackend.hh>


Second, hange the linear solver to the AMG solver 
from the AMGBackend

using LinearSolver = Dumux::AMGBackend<TypeTag>;

and recompile your application.

\subsection{Start parallel computation}
The parallel simulation is starte with mpirun followed by -np and
the number of cores that should be used and the executable. 

mpirun -np n_cores executable

On HPC cluster you usually have to use  qeuing system like (e.g. slurm). 



\subsection{Handling Parallel Results}
The results sould not differ between parallel an serial execution. As in
the serial case you get vtu-files as output. However you have an additional
variable "process rank" that shows the processor rank of each MPI partition.








