\chapter*{Discussion about the new handbook}
\todo[inline]{Ihr konnt die todo Befehle benutzen um Dinge zu markieren, die euch
  auffallen, dann können wir die beim nachsten Mal besprechen}
\section*{Beschlossene ToDos}
DEADLINE: 29.07.15
\begin{itemize}
  \item Modelle raus, Liste rein (Christoph)
  \item Neues Gitterkapitel (Natalie)
  \begin{itemize}
    \item gmesh (Timo)
    \item petrel (Alex)
    \item[x] artgridcreator (Nicolas)
    \item[x] dgf (kurz kommentieren)
    \item[x] icemcfd (nicolas)
  \end{itemize}
  \item Wiki
  \begin{itemize}
    \item[+] behandla, Mailing list (LH2), external Modules, lectures, tests
             teilweise schon erledigt -$>$ Rest (Vishal)
    \item[+] Infos fur neue Doktoranden (Christoph)
  \end{itemize}
  \item Hinweis auf tests, lecture und feature list am Ende von Tutorial
        (Gaby)
  \item[x] Doxygen (main page - link auf modules, featureList und parameterList)
        (Kilian)
  \item Newton etwas ausfuhrlicher und Dumux spezifischer, schematische
        Skizze der Matrix (Christoph)
  \item[x] Kapitel 4 weiter aufräumen (Christoph+Thomas)
  \begin{itemize}
    \item Tips und Tricks weiter integrieren
    \item Reihenfolge der Unterkapitel
  \end{itemize}
  \item[x] Kapitel 5 weiter aufräumen (Christoph+Thomas)
  \begin{itemize}
    \item Reihenfolge der Unterkapitel
    \item \Dumux aus den Uberschriften in Kap 5 rausnehmen
  \end{itemize}
  \item[x] release manager tasks + tutorials anschauen/testen (Thomas)
  \item Rechtschreibung - pdf in word öffnen
\end{itemize}

\section*{Weitere ToDos}
\begin{itemize}
  \item Kapitel 5: Unterkapitel aufteilen -$>$ weiter aufräumen (s. Todos)
\end{itemize}

\section*{Offene Fragen}
\begin{itemize}
  \item Was kann ins Wiki ausgelagert werden?
  \item Was kann ins doxygen ausgelagert werden?
  \item Alles was (nicht?) LH2 spezifische raus?
  \item Fur listOfProperties und listOfFeatures auf doxygen verweisen! (kommt
        unter doxygen main pages)
  \item Modelbeschreibungen rausschmeißen?
  \item bennenung der ausgelagerten tex dateien mit kapitelnummer (nur eine nummer)
        beginnend
  \item Neue Kapitel/Abschnitte
  \begin{itemize}
    \item Wie kann ich Gitter erstellen (externe tools), /einlesen (dune), wo
          kann ich bei dune nachschauen. Welche Dateien werden uberhaupt unterstutzt?
          in Kap 5 (jemanden finden, der sich auskennt Alex, Timo, Bernd)
  \end{itemize}
  \item Tutorials:
  \begin{itemize}
    \item Noch aktuell, noch funktioniert?
    \item Welche Features, Modelle brauchten auch ein Tutorial? Wie konnen wir
          den Einstieg leicht machen und fur uns den Aufwand gering halten?
    \item Verweis auf lecture fur realistischere Anwendungen
  \end{itemize}
\end{itemize}

\section*{Erledigt}
\begin{itemize}
  \item leer
\end{itemize}
