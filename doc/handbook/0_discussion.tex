\chapter{Discussion about the new handbook}

BITTE: Fur svn ist es schrecklich eine lange Code-Zeile zu haben, bitte fugt
manuelle Zeilenumbruche hinzu
\todo[inline]{Ihr konnt die todo Befehle benutzen um Dinge zu markieren, die euch
  auffallen, dann können wir die beim nachsten Mal besprechen}
\todo[inline]{Bitte schaut mal ob die grobe Kapitelstrukutur stimmt, durch das
  umziehen der Dateien, kann sein, das ich etwas beim andern der chapter, sections, etc
  vergessen habe. Kann sein, das es dadurch auch etwas unübersichtlich wurde, da
  subsubsection zu paragraphs wurden. Wir sollten vllt die Kapitel aufteilen und
  dann jeder eines durchgehen.}
\section{Beschlossene ToDos}
\begin{itemize}
  \item Umstrukturierung (Thomas)
  \item \Dumux aus den Uberschriften in Kap 5 rausnehmen
  \item Modelle raus, Liste rein (Christoph)
  \item Neues Gitterkapitel (Natalie)
  \item Wiki (behandla, Mailing list (LH2), external Modules, lectures, tests
        (Vishal), Infos fur neue Doktoranden (Christoph)
  \item Hinweis auf tests, lecture und feature list am Ende von Tutorial
  \item Doxygen main page - feature list und parameter list
  \item Newton etwas ausfuhrlicher und Dumux spezifischer, schematische
        Skizze der Matrix (Christoph)
\end{itemize}

\section{Weitere ToDos}
\begin{itemize}
  \item Wenn wir einen guten Weg zum Rechtschreibung prüfen finden, ware das sicher
        auch gut.
  \item Reihenfolge der Unterkapitel
  \item Name fur Kapitel 4 und 5
  \item man sieht kaum einen Unterschied zwischen subsubsection und paragraphs s.u.
  \item wir haben listtings für bash code (ist OK), für c++ und dumux code (sollten
        wir schauen wo was benutzt wird und den dumux code entfernen, außerdem wären
        Farben ganz nett), dazu könnten wir dann noch ein dumuxInputFile listing
        mit anderer Farbgebung (da andere Kommentare) erstellen
\end{itemize}

\subsubsection{subsubsection}
tafsasf mskmf safijsafaSF SAKFMAÖLSFMLDSF 

\paragraph{paragraph}
tafsasf mskmf safijsafaSF SAKFMAÖLSFMLDSF 

\section{Open Questions}
\begin{itemize}
  \item Was kann ins Wiki ausgelagert werden?
  \item Was kann ins doxygen ausgelagert werden?
  \item Alles was (nicht?) LH2 spezifische raus?
  \item Fur listOfProperties und listOfFeatures auf doxygen verweisen! (kommt
        unter doxygen main pages)
  \item Modelbeschreibungen rausschmeißen?
  \item bennenung der ausgelagerten tex dateien mit kapitelnummer (nur eine nummer)
        beginnend
  \item Neue Kapitel/Abschnitte
  \begin{itemize}
    \item Wie kann ich Gitter erstellen (externe tools), /einlesen (dune), wo
          kann ich bei dune nachschauen. Welche Dateien werden uberhaupt unterstutzt?
          in Kap 5 (jemanden finden, der sich auskennt Alex, Timo, Bernd)
  \end{itemize}
  \item Tutorials:
  \begin{itemize}
    \item Noch aktuell, noch funktioniert?
    \item Welche Features, Modelle brauchten auch ein Tutorial? Wie konnen wir
          den Einstieg leicht machen und fur uns den Aufwand gering halten?
    \item Verweis auf lecture fur realistischere Anwendungen
  \end{itemize}
\end{itemize}

\section{Other}
\subsection{How to move things to stable}
\begin{itemize}
  \item mandatory checks
  \begin{itemize}
    \item remove all warnings (compile e.g. with the pedantic option)
    \item are all units given?
    \item are all references .g. for model constant, fluids ... given?
    \item no tabulators?
    \item no trailing whitespaces?
    \item is the gnu license still up to date?
    \item valgrind?
    \item run make headercheck
  \end{itemize}
  \item optional checks
  \begin{itemize}
    \item check whether the todos are necessary
    \item which lines of code may lead to confusion -> comment them?
    \item are the comments necessary and helpful?
    \item check and implement the naming conventions from dumux-devel/doc/naminglist
    \item are the svn ignore properties set correctly?
  \end{itemize}
  \item doxygen (mandatory)
  \begin{itemize}
    \item fill all please doc me placeholders
    \item are all public function commented? (minimum level: brief documentation and
          explanation of the function parameters)
    \item compile doxygen
  \end{itemize}
  \item for new models
  \begin{itemize}
    \item no preprocessor macros on the model level
    \item integrate the model description into the handbook
    \item add a new section in doxygen via the modules.txt
    \item check that all equations appear correctly in doxygen
    \item check that all class definition are listed, are they also listed in
          other groups, where they may be necessary (e.g. fluxvariables)
    \item add suitable test problem(s) that tests the main features
  \end{itemize}
  \item for new problems
  \begin{itemize}
    \item add a descriptive problem description (e.g. size, BC, what can you see,
          where is the example taken from)
    \item add reference solution and include in automatic testing
    \item are the input files cleaned up (no unnecessary parameters, are the units given,
          no unnecessary comments, ...) maybe compare to already existing input files
    \item can you see the main features if you run the reference problem
  \end{itemize}
\end{itemize}